
% Default to the notebook output style

    


% Inherit from the specified cell style.




    
\documentclass{article}

    
    
    \usepackage{graphicx} % Used to insert images
    \usepackage{adjustbox} % Used to constrain images to a maximum size 
    \usepackage{color} % Allow colors to be defined
    \usepackage{enumerate} % Needed for markdown enumerations to work
    \usepackage{geometry} % Used to adjust the document margins
    \usepackage{amsmath} % Equations
    \usepackage{amssymb} % Equations
    \usepackage{eurosym} % defines \euro
    \usepackage[mathletters]{ucs} % Extended unicode (utf-8) support
    \usepackage[utf8x]{inputenc} % Allow utf-8 characters in the tex document
    \usepackage{fancyvrb} % verbatim replacement that allows latex
    \usepackage{grffile} % extends the file name processing of package graphics 
                         % to support a larger range 
    % The hyperref package gives us a pdf with properly built
    % internal navigation ('pdf bookmarks' for the table of contents,
    % internal cross-reference links, web links for URLs, etc.)
    \usepackage{hyperref}
    \usepackage{longtable} % longtable support required by pandoc >1.10
    \usepackage{booktabs}  % table support for pandoc > 1.12.2
    \usepackage{ulem} % ulem is needed to support strikethroughs (\sout)
    

    
    
    \definecolor{orange}{cmyk}{0,0.4,0.8,0.2}
    \definecolor{darkorange}{rgb}{.71,0.21,0.01}
    \definecolor{darkgreen}{rgb}{.12,.54,.11}
    \definecolor{myteal}{rgb}{.26, .44, .56}
    \definecolor{gray}{gray}{0.45}
    \definecolor{lightgray}{gray}{.95}
    \definecolor{mediumgray}{gray}{.8}
    \definecolor{inputbackground}{rgb}{.95, .95, .85}
    \definecolor{outputbackground}{rgb}{.95, .95, .95}
    \definecolor{traceback}{rgb}{1, .95, .95}
    % ansi colors
    \definecolor{red}{rgb}{.6,0,0}
    \definecolor{green}{rgb}{0,.65,0}
    \definecolor{brown}{rgb}{0.6,0.6,0}
    \definecolor{blue}{rgb}{0,.145,.698}
    \definecolor{purple}{rgb}{.698,.145,.698}
    \definecolor{cyan}{rgb}{0,.698,.698}
    \definecolor{lightgray}{gray}{0.5}
    
    % bright ansi colors
    \definecolor{darkgray}{gray}{0.25}
    \definecolor{lightred}{rgb}{1.0,0.39,0.28}
    \definecolor{lightgreen}{rgb}{0.48,0.99,0.0}
    \definecolor{lightblue}{rgb}{0.53,0.81,0.92}
    \definecolor{lightpurple}{rgb}{0.87,0.63,0.87}
    \definecolor{lightcyan}{rgb}{0.5,1.0,0.83}
    
    % commands and environments needed by pandoc snippets
    % extracted from the output of `pandoc -s`
    \providecommand{\tightlist}{%
      \setlength{\itemsep}{0pt}\setlength{\parskip}{0pt}}
    \DefineVerbatimEnvironment{Highlighting}{Verbatim}{commandchars=\\\{\}}
    % Add ',fontsize=\small' for more characters per line
    \newenvironment{Shaded}{}{}
    \newcommand{\KeywordTok}[1]{\textcolor[rgb]{0.00,0.44,0.13}{\textbf{{#1}}}}
    \newcommand{\DataTypeTok}[1]{\textcolor[rgb]{0.56,0.13,0.00}{{#1}}}
    \newcommand{\DecValTok}[1]{\textcolor[rgb]{0.25,0.63,0.44}{{#1}}}
    \newcommand{\BaseNTok}[1]{\textcolor[rgb]{0.25,0.63,0.44}{{#1}}}
    \newcommand{\FloatTok}[1]{\textcolor[rgb]{0.25,0.63,0.44}{{#1}}}
    \newcommand{\CharTok}[1]{\textcolor[rgb]{0.25,0.44,0.63}{{#1}}}
    \newcommand{\StringTok}[1]{\textcolor[rgb]{0.25,0.44,0.63}{{#1}}}
    \newcommand{\CommentTok}[1]{\textcolor[rgb]{0.38,0.63,0.69}{\textit{{#1}}}}
    \newcommand{\OtherTok}[1]{\textcolor[rgb]{0.00,0.44,0.13}{{#1}}}
    \newcommand{\AlertTok}[1]{\textcolor[rgb]{1.00,0.00,0.00}{\textbf{{#1}}}}
    \newcommand{\FunctionTok}[1]{\textcolor[rgb]{0.02,0.16,0.49}{{#1}}}
    \newcommand{\RegionMarkerTok}[1]{{#1}}
    \newcommand{\ErrorTok}[1]{\textcolor[rgb]{1.00,0.00,0.00}{\textbf{{#1}}}}
    \newcommand{\NormalTok}[1]{{#1}}
    
    % Additional commands for more recent versions of Pandoc
    \newcommand{\ConstantTok}[1]{\textcolor[rgb]{0.53,0.00,0.00}{{#1}}}
    \newcommand{\SpecialCharTok}[1]{\textcolor[rgb]{0.25,0.44,0.63}{{#1}}}
    \newcommand{\VerbatimStringTok}[1]{\textcolor[rgb]{0.25,0.44,0.63}{{#1}}}
    \newcommand{\SpecialStringTok}[1]{\textcolor[rgb]{0.73,0.40,0.53}{{#1}}}
    \newcommand{\ImportTok}[1]{{#1}}
    \newcommand{\DocumentationTok}[1]{\textcolor[rgb]{0.73,0.13,0.13}{\textit{{#1}}}}
    \newcommand{\AnnotationTok}[1]{\textcolor[rgb]{0.38,0.63,0.69}{\textbf{\textit{{#1}}}}}
    \newcommand{\CommentVarTok}[1]{\textcolor[rgb]{0.38,0.63,0.69}{\textbf{\textit{{#1}}}}}
    \newcommand{\VariableTok}[1]{\textcolor[rgb]{0.10,0.09,0.49}{{#1}}}
    \newcommand{\ControlFlowTok}[1]{\textcolor[rgb]{0.00,0.44,0.13}{\textbf{{#1}}}}
    \newcommand{\OperatorTok}[1]{\textcolor[rgb]{0.40,0.40,0.40}{{#1}}}
    \newcommand{\BuiltInTok}[1]{{#1}}
    \newcommand{\ExtensionTok}[1]{{#1}}
    \newcommand{\PreprocessorTok}[1]{\textcolor[rgb]{0.74,0.48,0.00}{{#1}}}
    \newcommand{\AttributeTok}[1]{\textcolor[rgb]{0.49,0.56,0.16}{{#1}}}
    \newcommand{\InformationTok}[1]{\textcolor[rgb]{0.38,0.63,0.69}{\textbf{\textit{{#1}}}}}
    \newcommand{\WarningTok}[1]{\textcolor[rgb]{0.38,0.63,0.69}{\textbf{\textit{{#1}}}}}
    
    
    % Define a nice break command that doesn't care if a line doesn't already
    % exist.
    \def\br{\hspace*{\fill} \\* }
    % Math Jax compatability definitions
    \def\gt{>}
    \def\lt{<}
    % Document parameters
    \title{ME596 Homework 5}
    
    
    

    % Pygments definitions
    
\makeatletter
\def\PY@reset{\let\PY@it=\relax \let\PY@bf=\relax%
    \let\PY@ul=\relax \let\PY@tc=\relax%
    \let\PY@bc=\relax \let\PY@ff=\relax}
\def\PY@tok#1{\csname PY@tok@#1\endcsname}
\def\PY@toks#1+{\ifx\relax#1\empty\else%
    \PY@tok{#1}\expandafter\PY@toks\fi}
\def\PY@do#1{\PY@bc{\PY@tc{\PY@ul{%
    \PY@it{\PY@bf{\PY@ff{#1}}}}}}}
\def\PY#1#2{\PY@reset\PY@toks#1+\relax+\PY@do{#2}}

\expandafter\def\csname PY@tok@kd\endcsname{\let\PY@bf=\textbf\def\PY@tc##1{\textcolor[rgb]{0.00,0.50,0.00}{##1}}}
\expandafter\def\csname PY@tok@gu\endcsname{\let\PY@bf=\textbf\def\PY@tc##1{\textcolor[rgb]{0.50,0.00,0.50}{##1}}}
\expandafter\def\csname PY@tok@gd\endcsname{\def\PY@tc##1{\textcolor[rgb]{0.63,0.00,0.00}{##1}}}
\expandafter\def\csname PY@tok@gr\endcsname{\def\PY@tc##1{\textcolor[rgb]{1.00,0.00,0.00}{##1}}}
\expandafter\def\csname PY@tok@o\endcsname{\def\PY@tc##1{\textcolor[rgb]{0.40,0.40,0.40}{##1}}}
\expandafter\def\csname PY@tok@s1\endcsname{\def\PY@tc##1{\textcolor[rgb]{0.73,0.13,0.13}{##1}}}
\expandafter\def\csname PY@tok@sc\endcsname{\def\PY@tc##1{\textcolor[rgb]{0.73,0.13,0.13}{##1}}}
\expandafter\def\csname PY@tok@nn\endcsname{\let\PY@bf=\textbf\def\PY@tc##1{\textcolor[rgb]{0.00,0.00,1.00}{##1}}}
\expandafter\def\csname PY@tok@na\endcsname{\def\PY@tc##1{\textcolor[rgb]{0.49,0.56,0.16}{##1}}}
\expandafter\def\csname PY@tok@sb\endcsname{\def\PY@tc##1{\textcolor[rgb]{0.73,0.13,0.13}{##1}}}
\expandafter\def\csname PY@tok@mb\endcsname{\def\PY@tc##1{\textcolor[rgb]{0.40,0.40,0.40}{##1}}}
\expandafter\def\csname PY@tok@gt\endcsname{\def\PY@tc##1{\textcolor[rgb]{0.00,0.27,0.87}{##1}}}
\expandafter\def\csname PY@tok@kc\endcsname{\let\PY@bf=\textbf\def\PY@tc##1{\textcolor[rgb]{0.00,0.50,0.00}{##1}}}
\expandafter\def\csname PY@tok@il\endcsname{\def\PY@tc##1{\textcolor[rgb]{0.40,0.40,0.40}{##1}}}
\expandafter\def\csname PY@tok@k\endcsname{\let\PY@bf=\textbf\def\PY@tc##1{\textcolor[rgb]{0.00,0.50,0.00}{##1}}}
\expandafter\def\csname PY@tok@gh\endcsname{\let\PY@bf=\textbf\def\PY@tc##1{\textcolor[rgb]{0.00,0.00,0.50}{##1}}}
\expandafter\def\csname PY@tok@vg\endcsname{\def\PY@tc##1{\textcolor[rgb]{0.10,0.09,0.49}{##1}}}
\expandafter\def\csname PY@tok@c\endcsname{\let\PY@it=\textit\def\PY@tc##1{\textcolor[rgb]{0.25,0.50,0.50}{##1}}}
\expandafter\def\csname PY@tok@s2\endcsname{\def\PY@tc##1{\textcolor[rgb]{0.73,0.13,0.13}{##1}}}
\expandafter\def\csname PY@tok@sh\endcsname{\def\PY@tc##1{\textcolor[rgb]{0.73,0.13,0.13}{##1}}}
\expandafter\def\csname PY@tok@bp\endcsname{\def\PY@tc##1{\textcolor[rgb]{0.00,0.50,0.00}{##1}}}
\expandafter\def\csname PY@tok@mh\endcsname{\def\PY@tc##1{\textcolor[rgb]{0.40,0.40,0.40}{##1}}}
\expandafter\def\csname PY@tok@m\endcsname{\def\PY@tc##1{\textcolor[rgb]{0.40,0.40,0.40}{##1}}}
\expandafter\def\csname PY@tok@si\endcsname{\let\PY@bf=\textbf\def\PY@tc##1{\textcolor[rgb]{0.73,0.40,0.53}{##1}}}
\expandafter\def\csname PY@tok@nb\endcsname{\def\PY@tc##1{\textcolor[rgb]{0.00,0.50,0.00}{##1}}}
\expandafter\def\csname PY@tok@nc\endcsname{\let\PY@bf=\textbf\def\PY@tc##1{\textcolor[rgb]{0.00,0.00,1.00}{##1}}}
\expandafter\def\csname PY@tok@sx\endcsname{\def\PY@tc##1{\textcolor[rgb]{0.00,0.50,0.00}{##1}}}
\expandafter\def\csname PY@tok@s\endcsname{\def\PY@tc##1{\textcolor[rgb]{0.73,0.13,0.13}{##1}}}
\expandafter\def\csname PY@tok@gi\endcsname{\def\PY@tc##1{\textcolor[rgb]{0.00,0.63,0.00}{##1}}}
\expandafter\def\csname PY@tok@err\endcsname{\def\PY@bc##1{\setlength{\fboxsep}{0pt}\fcolorbox[rgb]{1.00,0.00,0.00}{1,1,1}{\strut ##1}}}
\expandafter\def\csname PY@tok@vc\endcsname{\def\PY@tc##1{\textcolor[rgb]{0.10,0.09,0.49}{##1}}}
\expandafter\def\csname PY@tok@go\endcsname{\def\PY@tc##1{\textcolor[rgb]{0.53,0.53,0.53}{##1}}}
\expandafter\def\csname PY@tok@sr\endcsname{\def\PY@tc##1{\textcolor[rgb]{0.73,0.40,0.53}{##1}}}
\expandafter\def\csname PY@tok@cp\endcsname{\def\PY@tc##1{\textcolor[rgb]{0.74,0.48,0.00}{##1}}}
\expandafter\def\csname PY@tok@ow\endcsname{\let\PY@bf=\textbf\def\PY@tc##1{\textcolor[rgb]{0.67,0.13,1.00}{##1}}}
\expandafter\def\csname PY@tok@ge\endcsname{\let\PY@it=\textit}
\expandafter\def\csname PY@tok@mf\endcsname{\def\PY@tc##1{\textcolor[rgb]{0.40,0.40,0.40}{##1}}}
\expandafter\def\csname PY@tok@cm\endcsname{\let\PY@it=\textit\def\PY@tc##1{\textcolor[rgb]{0.25,0.50,0.50}{##1}}}
\expandafter\def\csname PY@tok@cs\endcsname{\let\PY@it=\textit\def\PY@tc##1{\textcolor[rgb]{0.25,0.50,0.50}{##1}}}
\expandafter\def\csname PY@tok@se\endcsname{\let\PY@bf=\textbf\def\PY@tc##1{\textcolor[rgb]{0.73,0.40,0.13}{##1}}}
\expandafter\def\csname PY@tok@kt\endcsname{\def\PY@tc##1{\textcolor[rgb]{0.69,0.00,0.25}{##1}}}
\expandafter\def\csname PY@tok@mo\endcsname{\def\PY@tc##1{\textcolor[rgb]{0.40,0.40,0.40}{##1}}}
\expandafter\def\csname PY@tok@ne\endcsname{\let\PY@bf=\textbf\def\PY@tc##1{\textcolor[rgb]{0.82,0.25,0.23}{##1}}}
\expandafter\def\csname PY@tok@cpf\endcsname{\let\PY@it=\textit\def\PY@tc##1{\textcolor[rgb]{0.25,0.50,0.50}{##1}}}
\expandafter\def\csname PY@tok@kr\endcsname{\let\PY@bf=\textbf\def\PY@tc##1{\textcolor[rgb]{0.00,0.50,0.00}{##1}}}
\expandafter\def\csname PY@tok@no\endcsname{\def\PY@tc##1{\textcolor[rgb]{0.53,0.00,0.00}{##1}}}
\expandafter\def\csname PY@tok@w\endcsname{\def\PY@tc##1{\textcolor[rgb]{0.73,0.73,0.73}{##1}}}
\expandafter\def\csname PY@tok@ni\endcsname{\let\PY@bf=\textbf\def\PY@tc##1{\textcolor[rgb]{0.60,0.60,0.60}{##1}}}
\expandafter\def\csname PY@tok@kp\endcsname{\def\PY@tc##1{\textcolor[rgb]{0.00,0.50,0.00}{##1}}}
\expandafter\def\csname PY@tok@sd\endcsname{\let\PY@it=\textit\def\PY@tc##1{\textcolor[rgb]{0.73,0.13,0.13}{##1}}}
\expandafter\def\csname PY@tok@c1\endcsname{\let\PY@it=\textit\def\PY@tc##1{\textcolor[rgb]{0.25,0.50,0.50}{##1}}}
\expandafter\def\csname PY@tok@nf\endcsname{\def\PY@tc##1{\textcolor[rgb]{0.00,0.00,1.00}{##1}}}
\expandafter\def\csname PY@tok@ss\endcsname{\def\PY@tc##1{\textcolor[rgb]{0.10,0.09,0.49}{##1}}}
\expandafter\def\csname PY@tok@gs\endcsname{\let\PY@bf=\textbf}
\expandafter\def\csname PY@tok@nv\endcsname{\def\PY@tc##1{\textcolor[rgb]{0.10,0.09,0.49}{##1}}}
\expandafter\def\csname PY@tok@mi\endcsname{\def\PY@tc##1{\textcolor[rgb]{0.40,0.40,0.40}{##1}}}
\expandafter\def\csname PY@tok@vi\endcsname{\def\PY@tc##1{\textcolor[rgb]{0.10,0.09,0.49}{##1}}}
\expandafter\def\csname PY@tok@kn\endcsname{\let\PY@bf=\textbf\def\PY@tc##1{\textcolor[rgb]{0.00,0.50,0.00}{##1}}}
\expandafter\def\csname PY@tok@nt\endcsname{\let\PY@bf=\textbf\def\PY@tc##1{\textcolor[rgb]{0.00,0.50,0.00}{##1}}}
\expandafter\def\csname PY@tok@gp\endcsname{\let\PY@bf=\textbf\def\PY@tc##1{\textcolor[rgb]{0.00,0.00,0.50}{##1}}}
\expandafter\def\csname PY@tok@ch\endcsname{\let\PY@it=\textit\def\PY@tc##1{\textcolor[rgb]{0.25,0.50,0.50}{##1}}}
\expandafter\def\csname PY@tok@nl\endcsname{\def\PY@tc##1{\textcolor[rgb]{0.63,0.63,0.00}{##1}}}
\expandafter\def\csname PY@tok@nd\endcsname{\def\PY@tc##1{\textcolor[rgb]{0.67,0.13,1.00}{##1}}}

\def\PYZbs{\char`\\}
\def\PYZus{\char`\_}
\def\PYZob{\char`\{}
\def\PYZcb{\char`\}}
\def\PYZca{\char`\^}
\def\PYZam{\char`\&}
\def\PYZlt{\char`\<}
\def\PYZgt{\char`\>}
\def\PYZsh{\char`\#}
\def\PYZpc{\char`\%}
\def\PYZdl{\char`\$}
\def\PYZhy{\char`\-}
\def\PYZsq{\char`\'}
\def\PYZdq{\char`\"}
\def\PYZti{\char`\~}
% for compatibility with earlier versions
\def\PYZat{@}
\def\PYZlb{[}
\def\PYZrb{]}
\makeatother


    % Exact colors from NB
    \definecolor{incolor}{rgb}{0.0, 0.0, 0.5}
    \definecolor{outcolor}{rgb}{0.545, 0.0, 0.0}



    
    % Prevent overflowing lines due to hard-to-break entities
    \sloppy 
    % Setup hyperref package
    \hypersetup{
      breaklinks=true,  % so long urls are correctly broken across lines
      colorlinks=true,
      urlcolor=blue,
      linkcolor=darkorange,
      citecolor=darkgreen,
      }
    % Slightly bigger margins than the latex defaults
    
    \geometry{verbose,tmargin=1in,bmargin=1in,lmargin=1in,rmargin=1in}
    \author{Erin Schmidt}
    \date{}
    

    \begin{document}
    
    
    \maketitle
    
    

    
    \subsection*{Problem 1.}\label{problem-1.}

Solve the following problem using the Simplex Search Method:

\[\mbox{min} \hspace{2 mm} f(x) = (1- x_1)^2 + (2 - x_2)^2\]

        \begin{Verbatim}[commandchars=\\\{\}]
{\color{incolor}In [{\color{incolor}2}]:} \PY{c+c1}{\PYZsh{}Simplex search}
        
        \PY{c+c1}{\PYZsh{}Erin Schmidt}
        
        \PY{c+c1}{\PYZsh{}for non\PYZhy{}linear programming problems ala Nelder and Mead(1965)}
        \PY{c+c1}{\PYZsh{}**Note** use Python3 for this script}
        \PY{k+kn}{import} \PY{n+nn}{math} \PY{k}{as} \PY{n+nn}{m}
        \PY{k+kn}{import} \PY{n+nn}{numpy} \PY{k}{as} \PY{n+nn}{np}
        
        \PY{k}{def} \PY{n+nf}{simplex\PYZus{}search}\PY{p}{(}\PY{n}{f}\PY{p}{,} \PY{n}{x\PYZus{}start}\PY{p}{,} \PY{n}{max\PYZus{}iter} \PY{o}{=} \PY{l+m+mi}{100}\PY{p}{,} \PY{n}{epsilon} \PY{o}{=} \PY{l+m+mi}{1}\PY{n}{E}\PY{o}{\PYZhy{}}\PY{l+m+mi}{6}\PY{p}{,} \PY{n}{gamma} \PY{o}{=} \PY{l+m+mi}{5}\PY{p}{,} \PY{n}{beta} \PY{o}{=} \PY{l+m+mf}{0.5}\PY{p}{)}\PY{p}{:}
            \PY{l+s+sd}{\PYZdq{}\PYZdq{}\PYZdq{}}
        \PY{l+s+sd}{    parameters of the function:}
        \PY{l+s+sd}{    f is the function to be optimized}
        \PY{l+s+sd}{    x\PYZus{}start (numpy array): initial position}
        \PY{l+s+sd}{    epsilon is the termination criteria}
        \PY{l+s+sd}{    gamma is the contraction coefficient}
        \PY{l+s+sd}{    beta is the expansion coefficient}
        \PY{l+s+sd}{    \PYZdq{}\PYZdq{}\PYZdq{}}
            
            \PY{c+c1}{\PYZsh{}init arrays}
            \PY{n}{N} \PY{o}{=} \PY{n+nb}{len}\PY{p}{(}\PY{n}{x\PYZus{}start}\PY{p}{)}
            \PY{n}{fnew} \PY{o}{=} \PY{p}{[}\PY{p}{]}
            \PY{n}{xnew} \PY{o}{=} \PY{p}{[}\PY{p}{]}
            \PY{n}{x} \PY{o}{=} \PY{p}{[}\PY{p}{]}
            
            \PY{c+c1}{\PYZsh{}generate vertices of initial simplex}
            \PY{n}{a} \PY{o}{=} \PY{o}{.}\PY{l+m+mi}{75}
            \PY{n}{x0} \PY{o}{=} \PY{p}{(}\PY{n}{x\PYZus{}start}\PY{p}{)}
            \PY{n}{x1} \PY{o}{=} \PY{p}{[}\PY{n}{x0} \PY{o}{+} \PY{p}{[}\PY{p}{(}\PY{p}{(}\PY{n}{N} \PY{o}{+} \PY{l+m+mi}{1}\PY{p}{)}\PY{o}{*}\PY{o}{*}\PY{l+m+mf}{0.5} \PY{o}{+} \PY{n}{N} \PY{o}{\PYZhy{}} \PY{l+m+mf}{1.}\PY{p}{)}\PY{o}{/}\PY{p}{(}\PY{n}{N} \PY{o}{+} \PY{l+m+mf}{1.}\PY{p}{)}\PY{o}{*}\PY{n}{a}\PY{p}{,} \PY{l+m+mf}{0.}\PY{p}{]}\PY{p}{]}
            \PY{n}{x2} \PY{o}{=} \PY{p}{[}\PY{n}{x0} \PY{o}{+} \PY{p}{[}\PY{l+m+mf}{0.}\PY{p}{,} \PY{p}{(}\PY{p}{(}\PY{n}{N} \PY{o}{+} \PY{l+m+mi}{1}\PY{p}{)}\PY{o}{*}\PY{o}{*}\PY{l+m+mf}{0.5} \PY{o}{\PYZhy{}} \PY{l+m+mf}{1.}\PY{p}{)}\PY{o}{/}\PY{p}{(}\PY{n}{N} \PY{o}{+} \PY{l+m+mf}{1.}\PY{p}{)}\PY{o}{*}\PY{n}{a}\PY{p}{]}\PY{p}{]} 
            \PY{n}{x3} \PY{o}{=} \PY{p}{[}\PY{n}{x0} \PY{o}{\PYZhy{}} \PY{p}{[}\PY{l+m+mf}{0.}\PY{p}{,} \PY{p}{(}\PY{p}{(}\PY{n}{N} \PY{o}{+} \PY{l+m+mi}{1}\PY{p}{)}\PY{o}{*}\PY{o}{*}\PY{l+m+mf}{0.5} \PY{o}{\PYZhy{}} \PY{l+m+mf}{1.}\PY{p}{)}\PY{o}{/}\PY{p}{(}\PY{n}{N} \PY{o}{+} \PY{l+m+mf}{1.}\PY{p}{)}\PY{o}{*}\PY{n}{a}\PY{p}{]}\PY{p}{]} 
            \PY{n}{x} \PY{o}{=} \PY{n}{np}\PY{o}{.}\PY{n}{vstack}\PY{p}{(}\PY{p}{(}\PY{n}{x1}\PY{p}{,} \PY{n}{x2}\PY{p}{,} \PY{n}{x3}\PY{p}{)}\PY{p}{)}
            \PY{c+c1}{\PYZsh{}print(x)}
        
            \PY{c+c1}{\PYZsh{}simplex iteration}
            \PY{k}{while} \PY{k+kc}{True}\PY{p}{:}
                \PY{c+c1}{\PYZsh{}find best, worst and 2nd worst points \PYZhy{}\PYZhy{}\PYZgt{} new center point}
                \PY{n}{f\PYZus{}run} \PY{o}{=} \PY{n}{np}\PY{o}{.}\PY{n}{array}\PY{p}{(}\PY{p}{[}\PY{n}{f}\PY{p}{(}\PY{n}{x}\PY{p}{[}\PY{l+m+mi}{0}\PY{p}{]}\PY{p}{)}\PY{p}{,} \PY{n}{f}\PY{p}{(}\PY{n}{x}\PY{p}{[}\PY{l+m+mi}{1}\PY{p}{]}\PY{p}{)}\PY{p}{,} \PY{n}{f}\PY{p}{(}\PY{n}{x}\PY{p}{[}\PY{l+m+mi}{2}\PY{p}{]}\PY{p}{)}\PY{p}{]}\PY{p}{)}\PY{o}{.}\PY{n}{tolist}\PY{p}{(}\PY{p}{)} \PY{c+c1}{\PYZsh{}func. values at vertices}
                \PY{c+c1}{\PYZsh{}print(f\PYZus{}run)}
                \PY{n}{xw} \PY{o}{=} \PY{n}{x}\PY{p}{[}\PY{n}{f\PYZus{}run}\PY{o}{.}\PY{n}{index}\PY{p}{(}\PY{n+nb}{sorted}\PY{p}{(}\PY{n}{f\PYZus{}run}\PY{p}{)}\PY{p}{[}\PY{o}{\PYZhy{}}\PY{l+m+mi}{1}\PY{p}{]}\PY{p}{)}\PY{p}{]}	\PY{c+c1}{\PYZsh{}worst point}
                \PY{n}{xb} \PY{o}{=} \PY{n}{x}\PY{p}{[}\PY{n}{f\PYZus{}run}\PY{o}{.}\PY{n}{index}\PY{p}{(}\PY{n+nb}{sorted}\PY{p}{(}\PY{n}{f\PYZus{}run}\PY{p}{)}\PY{p}{[}\PY{l+m+mi}{0}\PY{p}{]}\PY{p}{)}\PY{p}{]} 	\PY{c+c1}{\PYZsh{}best point}
                \PY{n}{xs} \PY{o}{=} \PY{n}{x}\PY{p}{[}\PY{n}{f\PYZus{}run}\PY{o}{.}\PY{n}{index}\PY{p}{(}\PY{n+nb}{sorted}\PY{p}{(}\PY{n}{f\PYZus{}run}\PY{p}{)}\PY{p}{[}\PY{o}{\PYZhy{}}\PY{l+m+mi}{2}\PY{p}{]}\PY{p}{)}\PY{p}{]}	\PY{c+c1}{\PYZsh{}2nd worst point}
                \PY{n}{xc} \PY{o}{=} \PY{p}{(}\PY{n}{xb} \PY{o}{+} \PY{n}{xs}\PY{p}{)}\PY{o}{/}\PY{n}{N} \PY{c+c1}{\PYZsh{}center point				  }
                \PY{n}{xr} \PY{o}{=} \PY{l+m+mi}{2}\PY{o}{*}\PY{n}{xc} \PY{o}{\PYZhy{}} \PY{n}{xw} \PY{c+c1}{\PYZsh{}reflection point}
                
                \PY{c+c1}{\PYZsh{}check cases}
                \PY{k}{if} \PY{n}{f}\PY{p}{(}\PY{n}{xr}\PY{p}{)} \PY{o}{\PYZlt{}} \PY{n}{f}\PY{p}{(}\PY{n}{xb}\PY{p}{)}\PY{p}{:} \PY{c+c1}{\PYZsh{}expansion}
                    \PY{n}{xnew} \PY{o}{=} \PY{l+m+mi}{2}\PY{o}{*}\PY{n}{xr} \PY{o}{\PYZhy{}} \PY{n}{xc}
                    \PY{c+c1}{\PYZsh{}xnew = (1 \PYZhy{} gamma)*xc \PYZhy{} gamma*xr}
                    \PY{c+c1}{\PYZsh{}print(\PYZsq{}a\PYZsq{}, f(xr), f(xb)) \PYZsh{}for debugging}
                \PY{k}{elif} \PY{n}{f}\PY{p}{(}\PY{n}{xr}\PY{p}{)} \PY{o}{\PYZgt{}} \PY{n}{f}\PY{p}{(}\PY{n}{xw}\PY{p}{)}\PY{p}{:} \PY{c+c1}{\PYZsh{}contraction 1}
                    \PY{n}{xnew} \PY{o}{=} \PY{p}{(}\PY{l+m+mi}{1} \PY{o}{\PYZhy{}} \PY{n}{beta}\PY{p}{)}\PY{o}{*}\PY{n}{xc} \PY{o}{+} \PY{n}{beta}\PY{o}{*}\PY{n}{xw}
                    \PY{c+c1}{\PYZsh{}print(\PYZsq{}b\PYZsq{}, f(xr), f(xw))}
                \PY{k}{elif} \PY{n}{f}\PY{p}{(}\PY{n}{xs}\PY{p}{)} \PY{o}{\PYZlt{}} \PY{n}{f}\PY{p}{(}\PY{n}{xr}\PY{p}{)} \PY{o+ow}{and} \PY{n}{f}\PY{p}{(}\PY{n}{xr}\PY{p}{)} \PY{o}{\PYZlt{}} \PY{n}{f}\PY{p}{(}\PY{n}{xw}\PY{p}{)}\PY{p}{:} \PY{c+c1}{\PYZsh{}contraction 2}
                    \PY{n}{xnew} \PY{o}{=} \PY{p}{(}\PY{l+m+mi}{1} \PY{o}{+} \PY{n}{beta}\PY{p}{)}\PY{o}{*}\PY{n}{xc} \PY{o}{\PYZhy{}} \PY{n}{beta}\PY{o}{*}\PY{n}{xw}
                    \PY{c+c1}{\PYZsh{}print(\PYZsq{}c\PYZsq{}, f(xs), f(xr), f(xw))}
                \PY{k}{else}\PY{p}{:}
                    \PY{n}{xnew} \PY{o}{=} \PY{n}{xr}
                
                \PY{c+c1}{\PYZsh{}replace vertices}
                \PY{l+s+sd}{\PYZdq{}\PYZdq{}\PYZdq{}}
        \PY{l+s+sd}{        if f(xnew) \PYZlt{} f(xb):}
        \PY{l+s+sd}{        \PYZdq{}\PYZdq{}\PYZdq{}}
                \PY{n}{x}\PY{p}{[}\PY{n}{f\PYZus{}run}\PY{o}{.}\PY{n}{index}\PY{p}{(}\PY{n+nb}{sorted}\PY{p}{(}\PY{n}{f\PYZus{}run}\PY{p}{)}\PY{p}{[}\PY{o}{\PYZhy{}}\PY{l+m+mi}{1}\PY{p}{]}\PY{p}{)}\PY{p}{]} \PY{o}{=} \PY{n}{xnew}
                \PY{c+c1}{\PYZsh{}x[1] = xb}
                \PY{c+c1}{\PYZsh{}x[2] = xs}
                \PY{n}{fnew}\PY{o}{.}\PY{n}{append}\PY{p}{(}\PY{n}{f}\PY{p}{(}\PY{n}{xnew}\PY{p}{)}\PY{p}{)}
                \PY{n+nb}{print}\PY{p}{(}\PY{l+s+s1}{\PYZsq{}}\PY{l+s+s1}{Current optimum = }\PY{l+s+s1}{\PYZsq{}}\PY{p}{,} \PY{n}{fnew}\PY{p}{[}\PY{o}{\PYZhy{}}\PY{l+m+mi}{1}\PY{p}{]}\PY{p}{)}
                
                \PY{c+c1}{\PYZsh{}break is any termination critera satisfied}
                \PY{k}{if} \PY{n+nb}{len}\PY{p}{(}\PY{n}{fnew}\PY{p}{)} \PY{o}{==} \PY{n}{max\PYZus{}iter} \PY{o+ow}{or} \PY{n}{term\PYZus{}check}\PY{p}{(}\PY{n}{xb}\PY{p}{,} \PY{n}{xc}\PY{p}{,} \PY{n}{xs}\PY{p}{,} \PY{n}{xnew}\PY{p}{,} \PY{n}{N}\PY{p}{)} \PY{o}{\PYZlt{}}\PY{o}{=} \PY{n}{epsilon}\PY{p}{:}
                    \PY{k}{return} \PY{p}{\PYZob{}}
                        \PY{n}{f}\PY{p}{(}\PY{n}{x}\PY{p}{[}\PY{n}{f\PYZus{}run}\PY{o}{.}\PY{n}{index}\PY{p}{(}\PY{n+nb}{sorted}\PY{p}{(}\PY{n}{f\PYZus{}run}\PY{p}{)}\PY{p}{[}\PY{l+m+mi}{0}\PY{p}{]}\PY{p}{)}\PY{p}{]}\PY{p}{)}\PY{p}{,} 
                        \PY{n}{x}\PY{p}{[}\PY{n}{f\PYZus{}run}\PY{o}{.}\PY{n}{index}\PY{p}{(}\PY{n+nb}{sorted}\PY{p}{(}\PY{n}{f\PYZus{}run}\PY{p}{)}\PY{p}{[}\PY{l+m+mi}{0}\PY{p}{]}\PY{p}{)}\PY{p}{]}\PY{p}{,} \PY{n+nb}{len}\PY{p}{(}\PY{n}{fnew}\PY{p}{)}
                        \PY{p}{\PYZcb{}}
        
        \PY{k}{def} \PY{n+nf}{term\PYZus{}check}\PY{p}{(}\PY{n}{xb}\PY{p}{,} \PY{n}{xc}\PY{p}{,} \PY{n}{xs}\PY{p}{,} \PY{n}{xnew}\PY{p}{,} \PY{n}{N}\PY{p}{)}\PY{p}{:} \PY{c+c1}{\PYZsh{}the termination critera}
            \PY{k}{return} \PY{n}{m}\PY{o}{.}\PY{n}{sqrt}\PY{p}{(}\PY{p}{(}\PY{p}{(}\PY{n}{f}\PY{p}{(}\PY{n}{xb}\PY{p}{)} \PY{o}{\PYZhy{}} \PY{n}{f}\PY{p}{(}\PY{n}{xc}\PY{p}{)}\PY{p}{)}\PY{o}{*}\PY{o}{*}\PY{l+m+mi}{2} \PY{o}{+} \PY{p}{(}\PY{n}{f}\PY{p}{(}\PY{n}{xnew}\PY{p}{)} \PY{o}{\PYZhy{}} \PY{n}{f}\PY{p}{(}\PY{n}{xc}\PY{p}{)}\PY{p}{)}\PY{o}{*}\PY{o}{*}\PY{l+m+mi}{2} \PY{o}{+} \PY{p}{(}\PY{n}{f}\PY{p}{(}\PY{n}{xs}\PY{p}{)} \PY{o}{\PYZhy{}} \PY{n}{f}\PY{p}{(}\PY{n}{xc}\PY{p}{)}\PY{p}{)}\PY{o}{*}\PY{o}{*}\PY{l+m+mi}{2}\PY{p}{)}\PY{o}{/}\PY{p}{(}\PY{n}{N} \PY{o}{+} \PY{l+m+mi}{1}\PY{p}{)}\PY{p}{)}
        
        \PY{c+c1}{\PYZsh{}testing}
        \PY{k}{def} \PY{n+nf}{f}\PY{p}{(}\PY{n}{z}\PY{p}{)}\PY{p}{:} \PY{c+c1}{\PYZsh{}the objective function}
            \PY{n}{x}\PY{o}{=}\PY{n}{z}
            \PY{k}{return} \PY{p}{(}\PY{l+m+mi}{1} \PY{o}{\PYZhy{}} \PY{n}{x}\PY{p}{[}\PY{l+m+mi}{0}\PY{p}{]}\PY{p}{)}\PY{o}{*}\PY{p}{(}\PY{l+m+mi}{1} \PY{o}{\PYZhy{}} \PY{n}{x}\PY{p}{[}\PY{l+m+mi}{0}\PY{p}{]}\PY{p}{)} \PY{o}{+} \PY{p}{(}\PY{l+m+mi}{2} \PY{o}{\PYZhy{}} \PY{n}{x}\PY{p}{[}\PY{l+m+mi}{1}\PY{p}{]}\PY{p}{)}\PY{o}{*}\PY{p}{(}\PY{l+m+mi}{2} \PY{o}{\PYZhy{}} \PY{n}{x}\PY{p}{[}\PY{l+m+mi}{1}\PY{p}{]}\PY{p}{)}
        
        \PY{c+c1}{\PYZsh{}print results}
        \PY{p}{(}\PY{n}{f}\PY{p}{,} \PY{n}{x}\PY{p}{,} \PY{n+nb}{iter}\PY{p}{)} \PY{o}{=} \PY{n}{simplex\PYZus{}search}\PY{p}{(}\PY{n}{f}\PY{p}{,} \PY{n}{np}\PY{o}{.}\PY{n}{array}\PY{p}{(}\PY{p}{[}\PY{l+m+mi}{0}\PY{p}{,}\PY{l+m+mi}{0}\PY{p}{]}\PY{p}{)}\PY{p}{)}
        \PY{c+c1}{\PYZsh{}print(\PYZsq{}\PYZbs{}n\PYZsq{})}
        \PY{n+nb}{print}\PY{p}{(}\PY{l+s+s1}{\PYZsq{}}\PY{l+s+s1}{f = }\PY{l+s+s1}{\PYZsq{}}\PY{p}{,} \PY{n}{f}\PY{p}{)}
        \PY{n+nb}{print}\PY{p}{(}\PY{l+s+s1}{\PYZsq{}}\PY{l+s+s1}{x = }\PY{l+s+s1}{\PYZsq{}}\PY{p}{,} \PY{n}{x}\PY{p}{)}
        \PY{n+nb}{print}\PY{p}{(}\PY{l+s+s1}{\PYZsq{}}\PY{l+s+s1}{iterations = }\PY{l+s+s1}{\PYZsq{}}\PY{p}{,} \PY{n+nb}{iter}\PY{p}{)}
\end{Verbatim}

    \begin{Verbatim}[commandchars=\\\{\}]
Current optimum =  1.84872055837
Current optimum =  2.87980947162
Current optimum =  1.91394746769
Current optimum =  0.132461620311
Current optimum =  1.27236172992
Current optimum =  0.516361455904
Current optimum =  0.139739849491
Current optimum =  0.0947731528218
Current optimum =  0.0218302612897
Current optimum =  0.00723261618504
Current optimum =  0.0234872486365
Current optimum =  0.00648150978763
Current optimum =  0.00501702904282
Current optimum =  8.01402718652e-05
Current optimum =  0.00356397186498
Current optimum =  0.00156201242228
Current optimum =  0.000579639268912
Current optimum =  0.000134229386486
Current optimum =  0.000111407397121
Current optimum =  2.44033741441e-05
Current optimum =  1.44168605522e-05
Current optimum =  7.24841787287e-06
Current optimum =  5.74226451148e-06
Current optimum =  2.61498507089e-06
Current optimum =  2.7872144543e-07
Current optimum =  6.31047342911e-07
Current optimum =  5.03785171206e-07
f =  2.7872144543e-07
x =  [ 1.00022415  1.99952201]
iterations =  27
    \end{Verbatim}

    \begin{Verbatim}[commandchars=\\\{\}]
{\color{incolor}In [{\color{incolor}3}]:} \PY{c+c1}{\PYZsh{}graphically verify the minimum}
        \PY{k+kn}{from} \PY{n+nn}{mpl\PYZus{}toolkits}\PY{n+nn}{.}\PY{n+nn}{mplot3d} \PY{k}{import} \PY{n}{Axes3D}
        \PY{k+kn}{from} \PY{n+nn}{matplotlib} \PY{k}{import} \PY{n}{cm}
        \PY{k+kn}{import} \PY{n+nn}{matplotlib}\PY{n+nn}{.}\PY{n+nn}{pyplot} \PY{k}{as} \PY{n+nn}{plt}
        \PY{o}{\PYZpc{}}\PY{k}{config} InlineBackend.figure\PYZus{}formats=[\PYZsq{}svg\PYZsq{}]
        \PY{o}{\PYZpc{}}\PY{k}{matplotlib} inline
        
        \PY{n}{x} \PY{o}{=} \PY{n}{np}\PY{o}{.}\PY{n}{arange}\PY{p}{(}\PY{o}{\PYZhy{}}\PY{l+m+mi}{10}\PY{p}{,}\PY{l+m+mi}{10}\PY{p}{,}\PY{o}{.}\PY{l+m+mi}{1}\PY{p}{)}
        \PY{n}{y} \PY{o}{=} \PY{n}{np}\PY{o}{.}\PY{n}{arange}\PY{p}{(}\PY{o}{\PYZhy{}}\PY{l+m+mi}{10}\PY{p}{,}\PY{l+m+mi}{10}\PY{p}{,}\PY{o}{.}\PY{l+m+mi}{1}\PY{p}{)}
        \PY{p}{(}\PY{n}{x}\PY{p}{,} \PY{n}{y}\PY{p}{)} \PY{o}{=} \PY{n}{np}\PY{o}{.}\PY{n}{meshgrid}\PY{p}{(}\PY{n}{x}\PY{p}{,} \PY{n}{y}\PY{p}{)}
        \PY{n}{z} \PY{o}{=} \PY{p}{(}\PY{l+m+mi}{1} \PY{o}{\PYZhy{}} \PY{n}{x}\PY{p}{)}\PY{o}{*}\PY{o}{*}\PY{l+m+mi}{2} \PY{o}{+} \PY{p}{(}\PY{l+m+mi}{2} \PY{o}{\PYZhy{}} \PY{n}{y}\PY{p}{)}\PY{o}{*}\PY{o}{*}\PY{l+m+mi}{2}
        \PY{n}{fig} \PY{o}{=} \PY{n}{plt}\PY{o}{.}\PY{n}{figure}\PY{p}{(}\PY{p}{)}
        \PY{n}{ax} \PY{o}{=} \PY{n}{fig}\PY{o}{.}\PY{n}{gca}\PY{p}{(}\PY{n}{projection}\PY{o}{=}\PY{l+s+s1}{\PYZsq{}}\PY{l+s+s1}{3d}\PY{l+s+s1}{\PYZsq{}}\PY{p}{)}
        \PY{n}{ax}\PY{o}{.}\PY{n}{plot\PYZus{}surface}\PY{p}{(}\PY{n}{x}\PY{p}{,} \PY{n}{y}\PY{p}{,} \PY{n}{z}\PY{p}{,} \PY{n}{label}\PY{o}{=}\PY{l+s+s1}{\PYZsq{}}\PY{l+s+s1}{parametric curve}\PY{l+s+s1}{\PYZsq{}}\PY{p}{,} \PY{n}{cmap}\PY{o}{=}\PY{n}{cm}\PY{o}{.}\PY{n}{jet}\PY{p}{,} \PY{n}{linewidth}\PY{o}{=}\PY{l+m+mf}{0.2}\PY{p}{)}
        \PY{c+c1}{\PYZsh{}ax.legend()}
        \PY{n}{plt}\PY{o}{.}\PY{n}{show}\PY{p}{(}\PY{p}{)}
\end{Verbatim}

    \begin{center}
    \adjustimage{max size={0.9\linewidth}{0.9\paperheight}}{ME596 Homework_5_files/ME596 Homework_5_2_0.pdf}
    \end{center}
    { \hspace*{\fill} \\}

    
    Quite sensibly the expansion and contraction coefficients, $\gamma$ and
$\beta$, should be larger than one and between zero and one
respectively. If values outside of this range are used, for instance by
using a $\beta$ value greater than one, and the search will march to the
extrema and return large function values, depending on the initial guess
of the design vector $x_0$. The parameter $\epsilon$ mostly determines
the precision of the final result, that is the number of decimals away
from the absolute optima point.

    \subsection*{Problem 2.}\label{problem-2}

Consider the constrained optimization problem

\[\mbox{Design variable} \, x = \left\{ \begin{array}{ll}
         x & \mbox{if $x \geq 0$};\\
        -x & \mbox{if $x < 0$}\end{array} \right. \]

\[ \mbox{min } f(x) = (x_1 - 3)^3 + (x_2 -3)^2\]

\[ \mbox{Subject to }\] \[\begin{array}{ll} x_1  &\leq 2 \\
                                        x_1x_2 &= 8 \end{array}\]

Based on an exterior penalty function method, employing $r_p = 5$
transform the problem into an unconstrained one by creating an
unconstrained pseudo-objective function $\Phi$. If a steepest descent
method is used to minimize $\Phi$ starting from the point $x = [3, 5]^T$
determine the search direction $d$ that should be used

Using an exterior penalty function our constraints should have the form

\[r_p(\mbox{max}(0, g_1)^2 + |g_2|^2).\]

Therefore the pseudo-objective function will have the form

\[\Phi = (x_1 - 3)^3 + (x_2 -3)^2 + r_p(\mbox{max}(0, x_1 - 2)^2 + (|x_1 x_2|- 2)^2).\]

Taking the gradient of the objective pseudo-function we have

\[\frac{\partial \Phi}{\partial x_1}  = \frac{\partial f}{\partial x_1} + r_p \left(2 g_1 \frac{\partial g_1}{\partial x_1} + 2g_2 \frac{\partial g_2}{\partial x_1} \right)\]

\[\frac{\partial \Phi}{\partial x_2} = \frac{\partial f}{\partial x_2} + r_p \left(2 g_1 \frac{\partial g_1}{\partial x_2} + 2g_2 \frac{\partial g_2}{\partial x_2} \right),\]

where

\begin{eqnarray*} \begin{split}
				 \frac{\partial f}{\partial x_1} &= 3(x_1 - 3)^2 \\
                \frac{\partial g_1}{\partial x_1} &= 1\\
                \frac{\partial g_2}{\partial x_1} &= |x_1x_2 - 8|x_2 \\ \end{split} \hspace{10 mm} \begin{split}
                \frac{\partial f}{\partial x_2} &= 2(x_2 -3) \\
                \frac{\partial g_1}{\partial x_2} &= 0 \\
                \frac{\partial g_2}{\partial x_2} &= |x_1x_2 -8|x_1. \end{split} \end{eqnarray*}.

Substituting up we have

\begin{eqnarray*} 
\frac{\partial \Phi}{\partial x_1} &=& 3(x_1 - 3)^2 =r_p(2 \mbox{max}(0, x_1-2) + 2|x_1x_2 -8|) \\
\frac{\partial \Phi}{\partial x_2} &=& 2(x_2 -3) +r_p(2|x_1x_2 - 8|x_2).
\end{eqnarray*}

The direction of steepest descent, $d$, is
$- \nabla \Phi = \langle \frac{\partial \Phi}{\partial x_1}, \frac{\partial \Phi}{\partial x_2} \rangle$.
At $x = [3,5]^T$ we can evaluate the direction of steepest decent as

\[\left. \frac{\partial \Phi}{\partial x_1}\right|_{x=(3,5)} = 5(2(1) + 2(7)(6)) = 360\]

\[\left. \frac{\partial \Phi}{\partial x_2}\right|_{x=(3,5)} = 2(2) + 5(2(7)(3)) = 218. \]

Normalizing by the gradient so that we have a unit vector, $d$, in the
direction of steepest descent

\[\frac{\langle 360, 218 \rangle}{\sqrt{360^2 + 218^2}} = \langle 0.855, 0.518 \rangle .\]

The next design point, assuming that a single variable search has been
performed to optimize the step size (yielding $\alpha = 0.002$), will be

\[x^{(i+1)} = x^{(i)} + \alpha_i d\]
\[x^{(i+1)} = x^{(i)} - \alpha_i \nabla \Phi\]

\begin{eqnarray*} x^{(i+1)} &=& (3,5) - 0.002 \langle 0.855, 0.518 \rangle \\
                         &=& (2.998, 4.99). \end{eqnarray*}

\subsection*{Addendum}\label{addendum}
The following is a summary and collection of notes of the presentation given by Dr. Christina Ivler, of the U.S. Army Research, Development and Engineering Command (RDECOM), during the April, 22nd seminar session.

\subsubsection{Dyanmics and Control of Fly-by-Wire Helicopters}
\begin{itemize}
\item There are engineering challenges pertaining to helicopter cargo ops.: a two-body dynamic system, non-collocated control problem which is "notoriously hard", there is no-direct control of the payload by the pilot.
\item Design goals: a control system with cable angle as an output. State-space coupling is key to the problem.
\item Tradeoffs: handling quality vs. load damping.
\item Other issues: pilot induced oscillations and feedback
\item Two constrained optimization operations performed on-the-fly (one for each mode): when the stick is active (e.g. the pilot is in control), and when the stick is fixed.
\item The optimization constraints include stability constraints, handling quality constraints, load damping constraints, minimize actuator activity.
\item System comparisons c.f. optimized baselines (e.g. other modern fly-by-wire systems): 1/2 the settling time for a 1000 kg test (but with only a single data point).
\end{itemize}
\subsubsection{Controls Allocation}
\begin{itemize}
\item The basic question is how to distribute control usage to get desired performance.
\item There are 4 basic allocation approaches:
\begin{enumerate}
\item Control Ganging
\item Cascaded Pseudo-Inverse
\item Linear Programming
\item Quadratic Programming
\end{enumerate}
\item The goal is to solve the control positions that give the desired moment $B_n = d$ where $B$ is $n \times m$ and $\dot{x} = Ax +B_n$.
\item There are issues with the legacy approaches to this problem
\begin{enumerate}
\item Pilot induced oscillations (more controls available than desired moments)
\item $B^T(BB^T)^{-1}$ minimizes $u^Tu$ if there is no saturation.
\end{enumerate}
\item New methods use convex optimization (e.g. min $J$, the cost function, subject to $B_n = d$ and \begin{eqnarray*}
u_{min} \leq &u& \leq u_{max}\\
\dot{u}_{min} \leq &\dot{u}& \leq \dot{u}_{max}
\end{eqnarray*}
\item The hard problem is doing the optimization in a real-time controller (64 Hz) with 50 iterations per time step.
\item Frequency sweeps were used to excite the system.
\item The optimization improved various metrics: phase delay, model following cost, percent time saturated, cost.
\item This approach \underline{does} have some drawbacks: the controller is non-deterministic, the optimization is computationally intensive.
\item During piloted test using quadratic programming control was reported to be "crisper" and with less over-shoot.
\end{itemize}
\subsubsection{UAV Research}
\begin{itemize}
\item Flight control challenges for UAV's: 
\begin{enumerate}
\item Poor models
\item Dynamics very unstable without feedback control
\item Control system design requirements poorly understood
\end{enumerate}
\item The research has useful applications, especially UAV control in GPS denied environments. 
\item Tried to identify frequency domains by exciting aircraft at a variety of frequencies.
\item Used dynamic inverse control laws with large bandwidth.
\end{itemize}
    % Add a bibliography block to the postdoc
    
    
    
    \end{document}
