
% Default to the notebook output style

    


% Inherit from the specified cell style.




    
\documentclass{article}

    
    
    \usepackage{graphicx} % Used to insert images
    \usepackage{adjustbox} % Used to constrain images to a maximum size 
    \usepackage{color} % Allow colors to be defined
    \usepackage{enumerate} % Needed for markdown enumerations to work
    \usepackage{geometry} % Used to adjust the document margins
    \usepackage{amsmath} % Equations
    \usepackage{amssymb} % Equations
    \usepackage{eurosym} % defines \euro
    \usepackage[mathletters]{ucs} % Extended unicode (utf-8) support
    \usepackage[utf8x]{inputenc} % Allow utf-8 characters in the tex document
    \usepackage{fancyvrb} % verbatim replacement that allows latex
    \usepackage{grffile} % extends the file name processing of package graphics 
                         % to support a larger range 
    % The hyperref package gives us a pdf with properly built
    % internal navigation ('pdf bookmarks' for the table of contents,
    % internal cross-reference links, web links for URLs, etc.)
    \usepackage{hyperref}
    \usepackage{longtable} % longtable support required by pandoc >1.10
    \usepackage{booktabs}  % table support for pandoc > 1.12.2
    \usepackage{ulem} % ulem is needed to support strikethroughs (\sout)
    

    
    
    \definecolor{orange}{cmyk}{0,0.4,0.8,0.2}
    \definecolor{darkorange}{rgb}{.71,0.21,0.01}
    \definecolor{darkgreen}{rgb}{.12,.54,.11}
    \definecolor{myteal}{rgb}{.26, .44, .56}
    \definecolor{gray}{gray}{0.45}
    \definecolor{lightgray}{gray}{.95}
    \definecolor{mediumgray}{gray}{.8}
    \definecolor{inputbackground}{rgb}{.95, .95, .85}
    \definecolor{outputbackground}{rgb}{.95, .95, .95}
    \definecolor{traceback}{rgb}{1, .95, .95}
    % ansi colors
    \definecolor{red}{rgb}{.6,0,0}
    \definecolor{green}{rgb}{0,.65,0}
    \definecolor{brown}{rgb}{0.6,0.6,0}
    \definecolor{blue}{rgb}{0,.145,.698}
    \definecolor{purple}{rgb}{.698,.145,.698}
    \definecolor{cyan}{rgb}{0,.698,.698}
    \definecolor{lightgray}{gray}{0.5}
    
    % bright ansi colors
    \definecolor{darkgray}{gray}{0.25}
    \definecolor{lightred}{rgb}{1.0,0.39,0.28}
    \definecolor{lightgreen}{rgb}{0.48,0.99,0.0}
    \definecolor{lightblue}{rgb}{0.53,0.81,0.92}
    \definecolor{lightpurple}{rgb}{0.87,0.63,0.87}
    \definecolor{lightcyan}{rgb}{0.5,1.0,0.83}
    
    % commands and environments needed by pandoc snippets
    % extracted from the output of `pandoc -s`
    \providecommand{\tightlist}{%
      \setlength{\itemsep}{0pt}\setlength{\parskip}{0pt}}
    \DefineVerbatimEnvironment{Highlighting}{Verbatim}{commandchars=\\\{\}}
    % Add ',fontsize=\small' for more characters per line
    \newenvironment{Shaded}{}{}
    \newcommand{\KeywordTok}[1]{\textcolor[rgb]{0.00,0.44,0.13}{\textbf{{#1}}}}
    \newcommand{\DataTypeTok}[1]{\textcolor[rgb]{0.56,0.13,0.00}{{#1}}}
    \newcommand{\DecValTok}[1]{\textcolor[rgb]{0.25,0.63,0.44}{{#1}}}
    \newcommand{\BaseNTok}[1]{\textcolor[rgb]{0.25,0.63,0.44}{{#1}}}
    \newcommand{\FloatTok}[1]{\textcolor[rgb]{0.25,0.63,0.44}{{#1}}}
    \newcommand{\CharTok}[1]{\textcolor[rgb]{0.25,0.44,0.63}{{#1}}}
    \newcommand{\StringTok}[1]{\textcolor[rgb]{0.25,0.44,0.63}{{#1}}}
    \newcommand{\CommentTok}[1]{\textcolor[rgb]{0.38,0.63,0.69}{\textit{{#1}}}}
    \newcommand{\OtherTok}[1]{\textcolor[rgb]{0.00,0.44,0.13}{{#1}}}
    \newcommand{\AlertTok}[1]{\textcolor[rgb]{1.00,0.00,0.00}{\textbf{{#1}}}}
    \newcommand{\FunctionTok}[1]{\textcolor[rgb]{0.02,0.16,0.49}{{#1}}}
    \newcommand{\RegionMarkerTok}[1]{{#1}}
    \newcommand{\ErrorTok}[1]{\textcolor[rgb]{1.00,0.00,0.00}{\textbf{{#1}}}}
    \newcommand{\NormalTok}[1]{{#1}}
    
    % Additional commands for more recent versions of Pandoc
    \newcommand{\ConstantTok}[1]{\textcolor[rgb]{0.53,0.00,0.00}{{#1}}}
    \newcommand{\SpecialCharTok}[1]{\textcolor[rgb]{0.25,0.44,0.63}{{#1}}}
    \newcommand{\VerbatimStringTok}[1]{\textcolor[rgb]{0.25,0.44,0.63}{{#1}}}
    \newcommand{\SpecialStringTok}[1]{\textcolor[rgb]{0.73,0.40,0.53}{{#1}}}
    \newcommand{\ImportTok}[1]{{#1}}
    \newcommand{\DocumentationTok}[1]{\textcolor[rgb]{0.73,0.13,0.13}{\textit{{#1}}}}
    \newcommand{\AnnotationTok}[1]{\textcolor[rgb]{0.38,0.63,0.69}{\textbf{\textit{{#1}}}}}
    \newcommand{\CommentVarTok}[1]{\textcolor[rgb]{0.38,0.63,0.69}{\textbf{\textit{{#1}}}}}
    \newcommand{\VariableTok}[1]{\textcolor[rgb]{0.10,0.09,0.49}{{#1}}}
    \newcommand{\ControlFlowTok}[1]{\textcolor[rgb]{0.00,0.44,0.13}{\textbf{{#1}}}}
    \newcommand{\OperatorTok}[1]{\textcolor[rgb]{0.40,0.40,0.40}{{#1}}}
    \newcommand{\BuiltInTok}[1]{{#1}}
    \newcommand{\ExtensionTok}[1]{{#1}}
    \newcommand{\PreprocessorTok}[1]{\textcolor[rgb]{0.74,0.48,0.00}{{#1}}}
    \newcommand{\AttributeTok}[1]{\textcolor[rgb]{0.49,0.56,0.16}{{#1}}}
    \newcommand{\InformationTok}[1]{\textcolor[rgb]{0.38,0.63,0.69}{\textbf{\textit{{#1}}}}}
    \newcommand{\WarningTok}[1]{\textcolor[rgb]{0.38,0.63,0.69}{\textbf{\textit{{#1}}}}}
    
    
    % Define a nice break command that doesn't care if a line doesn't already
    % exist.
    \def\br{\hspace*{\fill} \\* }
    % Math Jax compatability definitions
    \def\gt{>}
    \def\lt{<}
    % Document parameters
    \title{ME596 Homework 4}
    \date{}
    \author{Erin Schmidt}
    
    

    % Pygments definitions
    
\makeatletter
\def\PY@reset{\let\PY@it=\relax \let\PY@bf=\relax%
    \let\PY@ul=\relax \let\PY@tc=\relax%
    \let\PY@bc=\relax \let\PY@ff=\relax}
\def\PY@tok#1{\csname PY@tok@#1\endcsname}
\def\PY@toks#1+{\ifx\relax#1\empty\else%
    \PY@tok{#1}\expandafter\PY@toks\fi}
\def\PY@do#1{\PY@bc{\PY@tc{\PY@ul{%
    \PY@it{\PY@bf{\PY@ff{#1}}}}}}}
\def\PY#1#2{\PY@reset\PY@toks#1+\relax+\PY@do{#2}}

\expandafter\def\csname PY@tok@kp\endcsname{\def\PY@tc##1{\textcolor[rgb]{0.00,0.50,0.00}{##1}}}
\expandafter\def\csname PY@tok@mo\endcsname{\def\PY@tc##1{\textcolor[rgb]{0.40,0.40,0.40}{##1}}}
\expandafter\def\csname PY@tok@sh\endcsname{\def\PY@tc##1{\textcolor[rgb]{0.73,0.13,0.13}{##1}}}
\expandafter\def\csname PY@tok@nc\endcsname{\let\PY@bf=\textbf\def\PY@tc##1{\textcolor[rgb]{0.00,0.00,1.00}{##1}}}
\expandafter\def\csname PY@tok@no\endcsname{\def\PY@tc##1{\textcolor[rgb]{0.53,0.00,0.00}{##1}}}
\expandafter\def\csname PY@tok@nt\endcsname{\let\PY@bf=\textbf\def\PY@tc##1{\textcolor[rgb]{0.00,0.50,0.00}{##1}}}
\expandafter\def\csname PY@tok@cs\endcsname{\let\PY@it=\textit\def\PY@tc##1{\textcolor[rgb]{0.25,0.50,0.50}{##1}}}
\expandafter\def\csname PY@tok@bp\endcsname{\def\PY@tc##1{\textcolor[rgb]{0.00,0.50,0.00}{##1}}}
\expandafter\def\csname PY@tok@nd\endcsname{\def\PY@tc##1{\textcolor[rgb]{0.67,0.13,1.00}{##1}}}
\expandafter\def\csname PY@tok@cm\endcsname{\let\PY@it=\textit\def\PY@tc##1{\textcolor[rgb]{0.25,0.50,0.50}{##1}}}
\expandafter\def\csname PY@tok@gt\endcsname{\def\PY@tc##1{\textcolor[rgb]{0.00,0.27,0.87}{##1}}}
\expandafter\def\csname PY@tok@ss\endcsname{\def\PY@tc##1{\textcolor[rgb]{0.10,0.09,0.49}{##1}}}
\expandafter\def\csname PY@tok@kr\endcsname{\let\PY@bf=\textbf\def\PY@tc##1{\textcolor[rgb]{0.00,0.50,0.00}{##1}}}
\expandafter\def\csname PY@tok@vg\endcsname{\def\PY@tc##1{\textcolor[rgb]{0.10,0.09,0.49}{##1}}}
\expandafter\def\csname PY@tok@gu\endcsname{\let\PY@bf=\textbf\def\PY@tc##1{\textcolor[rgb]{0.50,0.00,0.50}{##1}}}
\expandafter\def\csname PY@tok@gi\endcsname{\def\PY@tc##1{\textcolor[rgb]{0.00,0.63,0.00}{##1}}}
\expandafter\def\csname PY@tok@nf\endcsname{\def\PY@tc##1{\textcolor[rgb]{0.00,0.00,1.00}{##1}}}
\expandafter\def\csname PY@tok@cpf\endcsname{\let\PY@it=\textit\def\PY@tc##1{\textcolor[rgb]{0.25,0.50,0.50}{##1}}}
\expandafter\def\csname PY@tok@kn\endcsname{\let\PY@bf=\textbf\def\PY@tc##1{\textcolor[rgb]{0.00,0.50,0.00}{##1}}}
\expandafter\def\csname PY@tok@sb\endcsname{\def\PY@tc##1{\textcolor[rgb]{0.73,0.13,0.13}{##1}}}
\expandafter\def\csname PY@tok@nv\endcsname{\def\PY@tc##1{\textcolor[rgb]{0.10,0.09,0.49}{##1}}}
\expandafter\def\csname PY@tok@go\endcsname{\def\PY@tc##1{\textcolor[rgb]{0.53,0.53,0.53}{##1}}}
\expandafter\def\csname PY@tok@gr\endcsname{\def\PY@tc##1{\textcolor[rgb]{1.00,0.00,0.00}{##1}}}
\expandafter\def\csname PY@tok@nb\endcsname{\def\PY@tc##1{\textcolor[rgb]{0.00,0.50,0.00}{##1}}}
\expandafter\def\csname PY@tok@kt\endcsname{\def\PY@tc##1{\textcolor[rgb]{0.69,0.00,0.25}{##1}}}
\expandafter\def\csname PY@tok@mi\endcsname{\def\PY@tc##1{\textcolor[rgb]{0.40,0.40,0.40}{##1}}}
\expandafter\def\csname PY@tok@c\endcsname{\let\PY@it=\textit\def\PY@tc##1{\textcolor[rgb]{0.25,0.50,0.50}{##1}}}
\expandafter\def\csname PY@tok@o\endcsname{\def\PY@tc##1{\textcolor[rgb]{0.40,0.40,0.40}{##1}}}
\expandafter\def\csname PY@tok@gh\endcsname{\let\PY@bf=\textbf\def\PY@tc##1{\textcolor[rgb]{0.00,0.00,0.50}{##1}}}
\expandafter\def\csname PY@tok@ow\endcsname{\let\PY@bf=\textbf\def\PY@tc##1{\textcolor[rgb]{0.67,0.13,1.00}{##1}}}
\expandafter\def\csname PY@tok@vi\endcsname{\def\PY@tc##1{\textcolor[rgb]{0.10,0.09,0.49}{##1}}}
\expandafter\def\csname PY@tok@kc\endcsname{\let\PY@bf=\textbf\def\PY@tc##1{\textcolor[rgb]{0.00,0.50,0.00}{##1}}}
\expandafter\def\csname PY@tok@ni\endcsname{\let\PY@bf=\textbf\def\PY@tc##1{\textcolor[rgb]{0.60,0.60,0.60}{##1}}}
\expandafter\def\csname PY@tok@vc\endcsname{\def\PY@tc##1{\textcolor[rgb]{0.10,0.09,0.49}{##1}}}
\expandafter\def\csname PY@tok@gd\endcsname{\def\PY@tc##1{\textcolor[rgb]{0.63,0.00,0.00}{##1}}}
\expandafter\def\csname PY@tok@sr\endcsname{\def\PY@tc##1{\textcolor[rgb]{0.73,0.40,0.53}{##1}}}
\expandafter\def\csname PY@tok@gs\endcsname{\let\PY@bf=\textbf}
\expandafter\def\csname PY@tok@ge\endcsname{\let\PY@it=\textit}
\expandafter\def\csname PY@tok@s\endcsname{\def\PY@tc##1{\textcolor[rgb]{0.73,0.13,0.13}{##1}}}
\expandafter\def\csname PY@tok@gp\endcsname{\let\PY@bf=\textbf\def\PY@tc##1{\textcolor[rgb]{0.00,0.00,0.50}{##1}}}
\expandafter\def\csname PY@tok@s2\endcsname{\def\PY@tc##1{\textcolor[rgb]{0.73,0.13,0.13}{##1}}}
\expandafter\def\csname PY@tok@c1\endcsname{\let\PY@it=\textit\def\PY@tc##1{\textcolor[rgb]{0.25,0.50,0.50}{##1}}}
\expandafter\def\csname PY@tok@err\endcsname{\def\PY@bc##1{\setlength{\fboxsep}{0pt}\fcolorbox[rgb]{1.00,0.00,0.00}{1,1,1}{\strut ##1}}}
\expandafter\def\csname PY@tok@k\endcsname{\let\PY@bf=\textbf\def\PY@tc##1{\textcolor[rgb]{0.00,0.50,0.00}{##1}}}
\expandafter\def\csname PY@tok@nn\endcsname{\let\PY@bf=\textbf\def\PY@tc##1{\textcolor[rgb]{0.00,0.00,1.00}{##1}}}
\expandafter\def\csname PY@tok@se\endcsname{\let\PY@bf=\textbf\def\PY@tc##1{\textcolor[rgb]{0.73,0.40,0.13}{##1}}}
\expandafter\def\csname PY@tok@s1\endcsname{\def\PY@tc##1{\textcolor[rgb]{0.73,0.13,0.13}{##1}}}
\expandafter\def\csname PY@tok@si\endcsname{\let\PY@bf=\textbf\def\PY@tc##1{\textcolor[rgb]{0.73,0.40,0.53}{##1}}}
\expandafter\def\csname PY@tok@ch\endcsname{\let\PY@it=\textit\def\PY@tc##1{\textcolor[rgb]{0.25,0.50,0.50}{##1}}}
\expandafter\def\csname PY@tok@mb\endcsname{\def\PY@tc##1{\textcolor[rgb]{0.40,0.40,0.40}{##1}}}
\expandafter\def\csname PY@tok@nl\endcsname{\def\PY@tc##1{\textcolor[rgb]{0.63,0.63,0.00}{##1}}}
\expandafter\def\csname PY@tok@w\endcsname{\def\PY@tc##1{\textcolor[rgb]{0.73,0.73,0.73}{##1}}}
\expandafter\def\csname PY@tok@sx\endcsname{\def\PY@tc##1{\textcolor[rgb]{0.00,0.50,0.00}{##1}}}
\expandafter\def\csname PY@tok@mf\endcsname{\def\PY@tc##1{\textcolor[rgb]{0.40,0.40,0.40}{##1}}}
\expandafter\def\csname PY@tok@m\endcsname{\def\PY@tc##1{\textcolor[rgb]{0.40,0.40,0.40}{##1}}}
\expandafter\def\csname PY@tok@cp\endcsname{\def\PY@tc##1{\textcolor[rgb]{0.74,0.48,0.00}{##1}}}
\expandafter\def\csname PY@tok@sc\endcsname{\def\PY@tc##1{\textcolor[rgb]{0.73,0.13,0.13}{##1}}}
\expandafter\def\csname PY@tok@sd\endcsname{\let\PY@it=\textit\def\PY@tc##1{\textcolor[rgb]{0.73,0.13,0.13}{##1}}}
\expandafter\def\csname PY@tok@na\endcsname{\def\PY@tc##1{\textcolor[rgb]{0.49,0.56,0.16}{##1}}}
\expandafter\def\csname PY@tok@ne\endcsname{\let\PY@bf=\textbf\def\PY@tc##1{\textcolor[rgb]{0.82,0.25,0.23}{##1}}}
\expandafter\def\csname PY@tok@il\endcsname{\def\PY@tc##1{\textcolor[rgb]{0.40,0.40,0.40}{##1}}}
\expandafter\def\csname PY@tok@kd\endcsname{\let\PY@bf=\textbf\def\PY@tc##1{\textcolor[rgb]{0.00,0.50,0.00}{##1}}}
\expandafter\def\csname PY@tok@mh\endcsname{\def\PY@tc##1{\textcolor[rgb]{0.40,0.40,0.40}{##1}}}

\def\PYZbs{\char`\\}
\def\PYZus{\char`\_}
\def\PYZob{\char`\{}
\def\PYZcb{\char`\}}
\def\PYZca{\char`\^}
\def\PYZam{\char`\&}
\def\PYZlt{\char`\<}
\def\PYZgt{\char`\>}
\def\PYZsh{\char`\#}
\def\PYZpc{\char`\%}
\def\PYZdl{\char`\$}
\def\PYZhy{\char`\-}
\def\PYZsq{\char`\'}
\def\PYZdq{\char`\"}
\def\PYZti{\char`\~}
% for compatibility with earlier versions
\def\PYZat{@}
\def\PYZlb{[}
\def\PYZrb{]}
\makeatother


    % Exact colors from NB
    \definecolor{incolor}{rgb}{0.0, 0.0, 0.5}
    \definecolor{outcolor}{rgb}{0.545, 0.0, 0.0}



    
    % Prevent overflowing lines due to hard-to-break entities
    \sloppy 
    % Setup hyperref package
    \hypersetup{
      breaklinks=true,  % so long urls are correctly broken across lines
      colorlinks=true,
      urlcolor=blue,
      linkcolor=darkorange,
      citecolor=darkgreen,
      }
    % Slightly bigger margins than the latex defaults
    
    \geometry{verbose,tmargin=1in,bmargin=1in,lmargin=1in,rmargin=1in}
    
    
    \begin{document}
    
    
    \maketitle
    
    

    
    \subsubsection*{Problem Statement}\label{problem-statement}

Maximum in-plane stress of a plane with a through-hole is given by

\[ \sigma = \frac{Kp}{(D-d)t},\]

where $t$ is the thickness of the plate, $p$ is the pressure applied,
and $K$ is the stress concentration factor. $K$ is given by

\[K = 1.11 +1.11\left(\frac{d}{D} \right)^{-0.18}.\]

We must find the hole size that minimizes the $\sigma$. We shall make
the notational simplification $\frac{d}{D} = x.$ We can note from the
problem formulation that we \emph{cannot} write $\sigma$ strictly in
terms of $x$; we must assume a value for $D$. We shall thence assume
$D$, as well as $p$, and $t$ are all equal to 1. Given our assumptions
we can write the stress equation as

\[\sigma (1 - x) = K.\]

This can be further simplified in terms of an explicit objective funtion
as

\[\sigma = \frac{1.11 + 1.11 x^{-0.18}}{1 - x}.\]

The objective function is plotted below.

    \begin{Verbatim}[commandchars=\\\{\}]
{\color{incolor}In [{\color{incolor}14}]:} \PY{k+kn}{import} \PY{n+nn}{numpy} \PY{k}{as} \PY{n+nn}{np}
         \PY{k+kn}{import} \PY{n+nn}{matplotlib}
         \PY{k+kn}{import} \PY{n+nn}{matplotlib}\PY{n+nn}{.}\PY{n+nn}{pyplot} \PY{k}{as} \PY{n+nn}{plt}
         \PY{k+kn}{from} \PY{n+nn}{\PYZus{}\PYZus{}future\PYZus{}\PYZus{}} \PY{k}{import} \PY{n}{division}
         \PY{o}{\PYZpc{}}\PY{k}{config} InlineBackend.figure\PYZus{}formats=[\PYZsq{}svg\PYZsq{}]
         \PY{o}{\PYZpc{}}\PY{k}{matplotlib} inline
         \PY{n}{plt}\PY{o}{.}\PY{n}{rc}\PY{p}{(}\PY{l+s+s1}{\PYZsq{}}\PY{l+s+s1}{pdf}\PY{l+s+s1}{\PYZsq{}}\PY{p}{,}\PY{n}{fonttype}\PY{o}{=}\PY{l+m+mi}{3}\PY{p}{)}          \PY{c+c1}{\PYZsh{} for proper subsetting of fonts}
         \PY{n}{plt}\PY{o}{.}\PY{n}{rc}\PY{p}{(}\PY{l+s+s1}{\PYZsq{}}\PY{l+s+s1}{axes}\PY{l+s+s1}{\PYZsq{}}\PY{p}{,}\PY{n}{linewidth}\PY{o}{=}\PY{l+m+mf}{0.5}\PY{p}{)}      \PY{c+c1}{\PYZsh{} thin axes; the default for lines is 1pt}
         \PY{n}{al} \PY{o}{=} \PY{n}{np}\PY{o}{.}\PY{n}{linspace}\PY{p}{(} \PY{l+m+mf}{0.05}\PY{p}{,} \PY{l+m+mf}{0.15}\PY{p}{,} \PY{l+m+mi}{500}\PY{p}{)}
         \PY{n}{plt}\PY{o}{.}\PY{n}{plot}\PY{p}{(}\PY{n}{al}\PY{p}{,} \PY{p}{(}\PY{l+m+mf}{1.11} \PY{o}{+} \PY{l+m+mf}{1.11}\PY{o}{*}\PY{n}{al}\PY{o}{*}\PY{o}{*}\PY{p}{(}\PY{o}{\PYZhy{}}\PY{l+m+mf}{0.18}\PY{p}{)}\PY{p}{)}\PY{o}{/}\PY{p}{(}\PY{l+m+mi}{1} \PY{o}{\PYZhy{}} \PY{n}{al}\PY{p}{)}\PY{p}{,} \PY{l+s+s1}{\PYZsq{}}\PY{l+s+s1}{k}\PY{l+s+s1}{\PYZsq{}}\PY{p}{)}
         \PY{n}{plt}\PY{o}{.}\PY{n}{axis}\PY{p}{(}\PY{p}{[}\PY{l+m+mf}{0.05}\PY{p}{,} \PY{l+m+mf}{0.15}\PY{p}{,} \PY{l+m+mf}{3.09}\PY{p}{,}\PY{l+m+mf}{3.17}\PY{p}{]}\PY{p}{)}
         \PY{n}{plt}\PY{o}{.}\PY{n}{title}\PY{p}{(}\PY{l+s+s2}{\PYZdq{}}\PY{l+s+s2}{Objective Function}\PY{l+s+s2}{\PYZdq{}}\PY{p}{)}
         \PY{n}{plt}\PY{o}{.}\PY{n}{ylabel}\PY{p}{(}\PY{l+s+s2}{\PYZdq{}}\PY{l+s+s2}{f}\PY{l+s+s2}{\PYZdq{}}\PY{p}{)}
         \PY{n}{plt}\PY{o}{.}\PY{n}{xlabel}\PY{p}{(}\PY{l+s+s2}{\PYZdq{}}\PY{l+s+s2}{x/D}\PY{l+s+s2}{\PYZdq{}}\PY{p}{)}
         \PY{n}{plt}\PY{o}{.}\PY{n}{show}\PY{p}{(}\PY{p}{)}
\end{Verbatim}

    \begin{center}
    \adjustimage{max size={0.9\linewidth}{0.9\paperheight}}{ME596 Homework 4_files/ME596 Homework 3_1_0.pdf}
    \end{center}
    { \hspace*{\fill} \\}
    
    We can see (at least qualitatively), from the plot of the objective
function that on the interval $0.05 < \alpha < 0.15$ the optimum value
lies somewhere between 0.09 and 0.10, and the function evaluated in that
range has an average value of about 3.10.

We shall proceed to find the minimum of the objective function by using
both an equal interval search algorithm and a polynomial approximation.

    \subsubsection*{Equal Interval Search}\label{equal-interval-search}

    \begin{Verbatim}[commandchars=\\\{\}]
{\color{incolor}In [{\color{incolor}18}]:} \PY{c+c1}{\PYZsh{}Equal Interval Search}
         \PY{c+c1}{\PYZsh{}Erin Schmidt}
         
         \PY{c+c1}{\PYZsh{}Adapted, with significant modification, from Arora et al.\PYZsq{}s APOLLO }
         \PY{c+c1}{\PYZsh{}implementation found in \PYZdq{}Introduction to Optimum Design\PYZdq{} 1st Ed. (1989).}
         
         \PY{k+kn}{import} \PY{n+nn}{numpy} \PY{k}{as} \PY{n+nn}{np}
         
         \PY{k}{def} \PY{n+nf}{func}\PY{p}{(}\PY{n}{al}\PY{p}{,} \PY{n}{count}\PY{p}{)}\PY{p}{:} \PY{c+c1}{\PYZsh{}the objective function}
             \PY{n}{count} \PY{o}{=} \PY{n}{count} \PY{o}{+} \PY{l+m+mi}{1}
             \PY{n}{f} \PY{o}{=} \PY{p}{(}\PY{l+m+mf}{1.11} \PY{o}{+} \PY{l+m+mf}{1.11}\PY{o}{*}\PY{n}{al}\PY{o}{*}\PY{o}{*}\PY{p}{(}\PY{o}{\PYZhy{}}\PY{l+m+mf}{0.18}\PY{p}{)}\PY{p}{)}\PY{o}{/}\PY{p}{(}\PY{l+m+mi}{1} \PY{o}{\PYZhy{}} \PY{n}{al}\PY{p}{)}
             \PY{k}{return} \PY{n}{f}\PY{p}{,} \PY{n}{count}
         
         \PY{k}{def} \PY{n+nf}{mini}\PY{p}{(}\PY{n}{au}\PY{p}{,} \PY{n}{al}\PY{p}{,} \PY{n}{count}\PY{p}{)}\PY{p}{:} \PY{c+c1}{\PYZsh{}evaluates f at the minimum (or optimum) stationary point}
             \PY{n}{alpha} \PY{o}{=} \PY{p}{(}\PY{n}{au} \PY{o}{+} \PY{n}{al}\PY{p}{)}\PY{o}{*}\PY{l+m+mf}{0.5}
             \PY{p}{(}\PY{n}{f}\PY{p}{,} \PY{n}{count}\PY{p}{)} \PY{o}{=} \PY{n}{func}\PY{p}{(}\PY{n}{alpha}\PY{p}{,} \PY{n}{count}\PY{p}{)}
             \PY{k}{return} \PY{n}{f}\PY{p}{,} \PY{n}{alpha}\PY{p}{,} \PY{n}{count}
         
         \PY{k}{def} \PY{n+nf}{equal}\PY{p}{(}\PY{n}{delta}\PY{p}{,} \PY{n}{epsilon}\PY{p}{,} \PY{n}{count}\PY{p}{,} \PY{n}{al}\PY{p}{)}\PY{p}{:}
             \PY{p}{(}\PY{n}{f}\PY{p}{,} \PY{n}{count}\PY{p}{)} \PY{o}{=} \PY{n}{func}\PY{p}{(}\PY{n}{al}\PY{p}{,} \PY{n}{count}\PY{p}{)}
             \PY{n}{fl} \PY{o}{=} \PY{n}{f} \PY{c+c1}{\PYZsh{}function value at lower bound}
             \PY{c+c1}{\PYZsh{}delta = 0.01 \PYZsh{}step\PYZhy{}size}
             \PY{c+c1}{\PYZsh{}au = 0.15 \PYZsh{}alpha upper bound}
         
             \PY{k}{while} \PY{k+kc}{True}\PY{p}{:}
                 \PY{n}{aa} \PY{o}{=} \PY{n}{delta}
                 \PY{p}{(}\PY{n}{f}\PY{p}{,} \PY{n}{count}\PY{p}{)} \PY{o}{=} \PY{n}{func}\PY{p}{(}\PY{n}{aa}\PY{p}{,} \PY{n}{count}\PY{p}{)}
                 \PY{n}{fa} \PY{o}{=} \PY{n}{f}
                 \PY{k}{if} \PY{n}{fa} \PY{o}{\PYZgt{}} \PY{n}{fl}\PY{p}{:}
                     \PY{n}{delta} \PY{o}{=} \PY{n}{delta} \PY{o}{*} \PY{l+m+mf}{0.1}
                 \PY{k}{else}\PY{p}{:}
                     \PY{k}{break}
                     
             \PY{k}{while} \PY{k+kc}{True}\PY{p}{:}
                 \PY{n}{au} \PY{o}{=} \PY{n}{aa} \PY{o}{+} \PY{n}{delta}
                 \PY{p}{(}\PY{n}{f}\PY{p}{,} \PY{n}{count}\PY{p}{)} \PY{o}{=} \PY{n}{func}\PY{p}{(}\PY{n}{au}\PY{p}{,} \PY{n}{count}\PY{p}{)}
                 \PY{n}{fu} \PY{o}{=} \PY{n}{f}
                 \PY{k}{if} \PY{n}{fa} \PY{o}{\PYZgt{}} \PY{n}{fu}\PY{p}{:}
                     \PY{n}{al} \PY{o}{=} \PY{n}{aa}
                     \PY{n}{aa} \PY{o}{=} \PY{n}{au}
                     \PY{n}{fl} \PY{o}{=} \PY{n}{fa}
                     \PY{n}{fa} \PY{o}{=} \PY{n}{fu}
                 \PY{k}{else}\PY{p}{:}
                     \PY{k}{break}
         
             \PY{k}{while} \PY{k+kc}{True}\PY{p}{:}
                 \PY{k}{if} \PY{p}{(}\PY{n}{au} \PY{o}{\PYZhy{}} \PY{n}{al}\PY{p}{)} \PY{o}{\PYZgt{}} \PY{n}{epsilon}\PY{p}{:} \PY{c+c1}{\PYZsh{}compares interval size to convergence criteria}
                     \PY{n}{delta} \PY{o}{=} \PY{n}{delta} \PY{o}{*} \PY{l+m+mf}{0.1}
                     \PY{n}{aa} \PY{o}{=} \PY{n}{al} \PY{c+c1}{\PYZsh{}intermediate alpha}
                     \PY{n}{fa} \PY{o}{=} \PY{n}{fl} \PY{c+c1}{\PYZsh{}intermediate alpha function value}
                     \PY{k}{while} \PY{k+kc}{True}\PY{p}{:}
                         \PY{n}{au} \PY{o}{=} \PY{n}{aa} \PY{o}{+} \PY{n}{delta}
                         \PY{p}{(}\PY{n}{f}\PY{p}{,} \PY{n}{count}\PY{p}{)} \PY{o}{=} \PY{n}{func}\PY{p}{(}\PY{n}{au}\PY{p}{,} \PY{n}{count}\PY{p}{)}
                         \PY{n}{fu} \PY{o}{=} \PY{n}{f}
                         \PY{k}{if} \PY{n}{fa} \PY{o}{\PYZgt{}} \PY{n}{fu}\PY{p}{:} 
                             \PY{n}{al} \PY{o}{=} \PY{n}{aa}
                             \PY{n}{aa} \PY{o}{=} \PY{n}{au}
                             \PY{n}{fl} \PY{o}{=} \PY{n}{fa}
                             \PY{n}{fa} \PY{o}{=} \PY{n}{fu}
                             \PY{k}{continue}
                         \PY{k}{else}\PY{p}{:}
                             \PY{k}{break}
                     \PY{k}{continue}
                 \PY{k}{else}\PY{p}{:}
                     \PY{p}{(}\PY{n}{f}\PY{p}{,} \PY{n}{alpha}\PY{p}{,} \PY{n}{count}\PY{p}{)} \PY{o}{=} \PY{n}{mini}\PY{p}{(}\PY{n}{au}\PY{p}{,} \PY{n}{al}\PY{p}{,} \PY{n}{count}\PY{p}{)}
                     \PY{k}{return} \PY{n}{f}\PY{p}{,} \PY{n}{alpha}\PY{p}{,} \PY{n}{count}
         
         
         \PY{c+c1}{\PYZsh{}run the program}
         \PY{n}{delta} \PY{o}{=} \PY{l+m+mf}{0.01}
         \PY{n}{epsilon} \PY{o}{=} \PY{l+m+mi}{1}\PY{n}{E}\PY{o}{\PYZhy{}}\PY{l+m+mi}{3}
         \PY{n}{count} \PY{o}{=} \PY{l+m+mi}{0}
         \PY{n}{al} \PY{o}{=} \PY{l+m+mf}{0.01} \PY{c+c1}{\PYZsh{} alpha lower bound}
         
         \PY{p}{(}\PY{n}{f}\PY{p}{,} \PY{n}{alpha}\PY{p}{,} \PY{n}{count}\PY{p}{)} \PY{o}{=} \PY{n}{equal}\PY{p}{(}\PY{n}{delta}\PY{p}{,} \PY{n}{epsilon}\PY{p}{,} \PY{n}{count}\PY{p}{,} \PY{n}{al}\PY{p}{)}
         \PY{n+nb}{print}\PY{p}{(}\PY{l+s+s1}{\PYZsq{}}\PY{l+s+s1}{The minimum is at }\PY{l+s+si}{\PYZob{}:.4f\PYZcb{}}\PY{l+s+s1}{\PYZsq{}}\PY{o}{.}\PY{n}{format}\PY{p}{(}\PY{n}{alpha}\PY{p}{)}\PY{p}{)}
         \PY{n+nb}{print}\PY{p}{(}\PY{l+s+s1}{\PYZsq{}}\PY{l+s+s1}{The function value at the minimum = }\PY{l+s+si}{\PYZob{}:.4f\PYZcb{}}\PY{l+s+s1}{\PYZsq{}}\PY{o}{.}\PY{n}{format}\PY{p}{(}\PY{n}{f}\PY{p}{)}\PY{p}{)}
         \PY{n+nb}{print}\PY{p}{(}\PY{l+s+s1}{\PYZsq{}}\PY{l+s+s1}{Total number of function calls = }\PY{l+s+si}{\PYZob{}\PYZcb{}}\PY{l+s+s1}{\PYZsq{}}\PY{o}{.}\PY{n}{format}\PY{p}{(}\PY{n}{count}\PY{p}{)}\PY{p}{)}
\end{Verbatim}

    \begin{Verbatim}[commandchars=\\\{\}]
The minimum is at 0.0979
The function value at the minimum = 3.1000
Total number of function calls = 32
    \end{Verbatim}

    \subsubsection*{Polynomial Approximation}\label{polynomial-approximation}

    \begin{Verbatim}[commandchars=\\\{\}]
{\color{incolor}In [{\color{incolor}22}]:} \PY{c+c1}{\PYZsh{} Polynomial approximation (4\PYZhy{}point cubic)}
         \PY{c+c1}{\PYZsh{} \PYZhy{}Erin Schmidt}
         
         \PY{k+kn}{import} \PY{n+nn}{numpy} \PY{k}{as} \PY{n+nn}{np}
         \PY{k+kn}{from} \PY{n+nn}{math} \PY{k}{import} \PY{n}{sqrt}
         
         \PY{c+c1}{\PYZsh{} make an array with random values between 0.05 and 0.15 with 4 entries}
         \PY{n}{x} \PY{o}{=} \PY{p}{(}\PY{l+m+mf}{0.05} \PY{o}{+} \PY{n}{np}\PY{o}{.}\PY{n}{random}\PY{o}{.}\PY{n}{sample}\PY{p}{(}\PY{l+m+mi}{4}\PY{p}{)}\PY{o}{*}\PY{l+m+mf}{0.15}\PY{p}{)}
         
         \PY{c+c1}{\PYZsh{} make an array of function values at the 4 points of x}
         \PY{k}{def} \PY{n+nf}{f}\PY{p}{(}\PY{n}{x}\PY{p}{)}\PY{p}{:} \PY{c+c1}{\PYZsh{} the objective function}
             \PY{k}{return} \PY{p}{(}\PY{l+m+mf}{1.11} \PY{o}{+} \PY{l+m+mf}{1.11}\PY{o}{*}\PY{n}{x}\PY{o}{*}\PY{o}{*}\PY{p}{(}\PY{o}{\PYZhy{}}\PY{l+m+mf}{0.18}\PY{p}{)}\PY{p}{)}\PY{o}{/}\PY{p}{(}\PY{l+m+mi}{1} \PY{o}{\PYZhy{}} \PY{n}{x}\PY{p}{)}
         
         \PY{n}{f\PYZus{}array} \PY{o}{=} \PY{p}{[}\PY{p}{]}
         \PY{n}{i} \PY{o}{=} \PY{l+m+mi}{0}
         \PY{k}{while} \PY{n}{i} \PY{o}{\PYZlt{}}\PY{o}{=} \PY{n+nb}{len}\PY{p}{(}\PY{n}{x}\PY{p}{)} \PY{o}{\PYZhy{}} \PY{l+m+mi}{1}\PY{p}{:}
             \PY{n}{f\PYZus{}array}\PY{o}{.}\PY{n}{append}\PY{p}{(}\PY{n}{f}\PY{p}{(}\PY{n}{x}\PY{p}{[}\PY{n}{i}\PY{p}{]}\PY{p}{)}\PY{p}{)}
             \PY{n}{i} \PY{o}{+}\PY{o}{=} \PY{l+m+mi}{1}
         
         \PY{c+c1}{\PYZsh{} use the equations from Vanderplaats 1984 to solve coefficients}
         \PY{n}{q1} \PY{o}{=} \PY{n}{x}\PY{p}{[}\PY{l+m+mi}{2}\PY{p}{]}\PY{o}{*}\PY{o}{*}\PY{l+m+mi}{3} \PY{o}{*} \PY{p}{(}\PY{n}{x}\PY{p}{[}\PY{l+m+mi}{1}\PY{p}{]} \PY{o}{\PYZhy{}} \PY{n}{x}\PY{p}{[}\PY{l+m+mi}{0}\PY{p}{]}\PY{p}{)} \PY{o}{\PYZhy{}} \PY{n}{x}\PY{p}{[}\PY{l+m+mi}{1}\PY{p}{]}\PY{o}{*}\PY{o}{*}\PY{l+m+mi}{3} \PY{o}{*} \PY{p}{(}\PY{n}{x}\PY{p}{[}\PY{l+m+mi}{2}\PY{p}{]} \PY{o}{\PYZhy{}} \PY{n}{x}\PY{p}{[}\PY{l+m+mi}{0}\PY{p}{]}\PY{p}{)} \PY{o}{+} \PY{n}{x}\PY{p}{[}\PY{l+m+mi}{0}\PY{p}{]}\PY{o}{*}\PY{o}{*}\PY{l+m+mi}{3} \PY{o}{*} \PY{p}{(}\PY{n}{x}\PY{p}{[}\PY{l+m+mi}{2}\PY{p}{]} \PY{o}{\PYZhy{}} \PY{n}{x}\PY{p}{[}\PY{l+m+mi}{1}\PY{p}{]}\PY{p}{)}
         \PY{n}{q2} \PY{o}{=} \PY{n}{x}\PY{p}{[}\PY{l+m+mi}{3}\PY{p}{]}\PY{o}{*}\PY{o}{*}\PY{l+m+mi}{3} \PY{o}{*} \PY{p}{(}\PY{n}{x}\PY{p}{[}\PY{l+m+mi}{1}\PY{p}{]} \PY{o}{\PYZhy{}} \PY{n}{x}\PY{p}{[}\PY{l+m+mi}{0}\PY{p}{]}\PY{p}{)} \PY{o}{\PYZhy{}} \PY{n}{x}\PY{p}{[}\PY{l+m+mi}{1}\PY{p}{]}\PY{o}{*}\PY{o}{*}\PY{l+m+mi}{3} \PY{o}{*} \PY{p}{(}\PY{n}{x}\PY{p}{[}\PY{l+m+mi}{3}\PY{p}{]} \PY{o}{\PYZhy{}} \PY{n}{x}\PY{p}{[}\PY{l+m+mi}{0}\PY{p}{]}\PY{p}{)} \PY{o}{+} \PY{n}{x}\PY{p}{[}\PY{l+m+mi}{0}\PY{p}{]}\PY{o}{*}\PY{o}{*}\PY{l+m+mi}{3} \PY{o}{*} \PY{p}{(}\PY{n}{x}\PY{p}{[}\PY{l+m+mi}{3}\PY{p}{]} \PY{o}{\PYZhy{}} \PY{n}{x}\PY{p}{[}\PY{l+m+mi}{1}\PY{p}{]}\PY{p}{)}
         \PY{n}{q3} \PY{o}{=} \PY{p}{(}\PY{n}{x}\PY{p}{[}\PY{l+m+mi}{2}\PY{p}{]} \PY{o}{\PYZhy{}} \PY{n}{x}\PY{p}{[}\PY{l+m+mi}{1}\PY{p}{]}\PY{p}{)} \PY{o}{*} \PY{p}{(}\PY{n}{x}\PY{p}{[}\PY{l+m+mi}{1}\PY{p}{]} \PY{o}{\PYZhy{}} \PY{n}{x}\PY{p}{[}\PY{l+m+mi}{0}\PY{p}{]}\PY{p}{)} \PY{o}{*} \PY{p}{(}\PY{n}{x}\PY{p}{[}\PY{l+m+mi}{2}\PY{p}{]} \PY{o}{\PYZhy{}} \PY{n}{x}\PY{p}{[}\PY{l+m+mi}{0}\PY{p}{]}\PY{p}{)}
         \PY{n}{q4} \PY{o}{=} \PY{p}{(}\PY{n}{x}\PY{p}{[}\PY{l+m+mi}{3}\PY{p}{]} \PY{o}{\PYZhy{}} \PY{n}{x}\PY{p}{[}\PY{l+m+mi}{1}\PY{p}{]}\PY{p}{)} \PY{o}{*} \PY{p}{(}\PY{n}{x}\PY{p}{[}\PY{l+m+mi}{1}\PY{p}{]} \PY{o}{\PYZhy{}} \PY{n}{x}\PY{p}{[}\PY{l+m+mi}{0}\PY{p}{]}\PY{p}{)} \PY{o}{*} \PY{p}{(}\PY{n}{x}\PY{p}{[}\PY{l+m+mi}{3}\PY{p}{]} \PY{o}{\PYZhy{}} \PY{n}{x}\PY{p}{[}\PY{l+m+mi}{0}\PY{p}{]}\PY{p}{)}
         \PY{n}{q5} \PY{o}{=} \PY{n}{f\PYZus{}array}\PY{p}{[}\PY{l+m+mi}{2}\PY{p}{]} \PY{o}{*} \PY{p}{(}\PY{n}{x}\PY{p}{[}\PY{l+m+mi}{1}\PY{p}{]} \PY{o}{\PYZhy{}} \PY{n}{x}\PY{p}{[}\PY{l+m+mi}{0}\PY{p}{]}\PY{p}{)} \PY{o}{\PYZhy{}} \PY{n}{f\PYZus{}array}\PY{p}{[}\PY{l+m+mi}{1}\PY{p}{]} \PY{o}{*} \PY{p}{(}\PY{n}{x}\PY{p}{[}\PY{l+m+mi}{2}\PY{p}{]} \PY{o}{\PYZhy{}} \PY{n}{x}\PY{p}{[}\PY{l+m+mi}{0}\PY{p}{]}\PY{p}{)} \PY{o}{+} \PY{n}{f\PYZus{}array}\PY{p}{[}\PY{l+m+mi}{0}\PY{p}{]} \PY{o}{*} \PY{p}{(}\PY{n}{x}\PY{p}{[}\PY{l+m+mi}{2}\PY{p}{]} \PY{o}{\PYZhy{}} \PY{n}{x}\PY{p}{[}\PY{l+m+mi}{1}\PY{p}{]}\PY{p}{)}
         \PY{n}{q6} \PY{o}{=} \PY{n}{f\PYZus{}array}\PY{p}{[}\PY{l+m+mi}{3}\PY{p}{]} \PY{o}{*} \PY{p}{(}\PY{n}{x}\PY{p}{[}\PY{l+m+mi}{1}\PY{p}{]} \PY{o}{\PYZhy{}} \PY{n}{x}\PY{p}{[}\PY{l+m+mi}{0}\PY{p}{]}\PY{p}{)} \PY{o}{\PYZhy{}} \PY{n}{f\PYZus{}array}\PY{p}{[}\PY{l+m+mi}{1}\PY{p}{]} \PY{o}{*} \PY{p}{(}\PY{n}{x}\PY{p}{[}\PY{l+m+mi}{3}\PY{p}{]} \PY{o}{\PYZhy{}} \PY{n}{x}\PY{p}{[}\PY{l+m+mi}{0}\PY{p}{]}\PY{p}{)} \PY{o}{+} \PY{n}{f\PYZus{}array}\PY{p}{[}\PY{l+m+mi}{0}\PY{p}{]} \PY{o}{*} \PY{p}{(}\PY{n}{x}\PY{p}{[}\PY{l+m+mi}{3}\PY{p}{]} \PY{o}{\PYZhy{}} \PY{n}{x}\PY{p}{[}\PY{l+m+mi}{1}\PY{p}{]}\PY{p}{)}
         
         \PY{n}{a3} \PY{o}{=} \PY{p}{(}\PY{n}{q3}\PY{o}{*}\PY{n}{q6} \PY{o}{\PYZhy{}} \PY{n}{q4}\PY{o}{*}\PY{n}{q5}\PY{p}{)}\PY{o}{/}\PY{p}{(}\PY{n}{q2}\PY{o}{*}\PY{n}{q3} \PY{o}{\PYZhy{}} \PY{n}{q1}\PY{o}{*}\PY{n}{q4}\PY{p}{)}
         \PY{n}{a2} \PY{o}{=} \PY{p}{(}\PY{n}{q5} \PY{o}{\PYZhy{}} \PY{n}{a3}\PY{o}{*}\PY{n}{q1}\PY{p}{)}\PY{o}{/}\PY{n}{q3}
         \PY{n}{a1} \PY{o}{=} \PY{p}{(}\PY{n}{f\PYZus{}array}\PY{p}{[}\PY{l+m+mi}{1}\PY{p}{]} \PY{o}{\PYZhy{}} \PY{n}{f\PYZus{}array}\PY{p}{[}\PY{l+m+mi}{0}\PY{p}{]}\PY{p}{)}\PY{o}{/}\PY{p}{(}\PY{n}{x}\PY{p}{[}\PY{l+m+mi}{1}\PY{p}{]} \PY{o}{\PYZhy{}} \PY{n}{x}\PY{p}{[}\PY{l+m+mi}{0}\PY{p}{]}\PY{p}{)} \PY{o}{\PYZhy{}} \PYZbs{}
         \PY{n}{a3}\PY{o}{*}\PY{p}{(}\PY{n}{x}\PY{p}{[}\PY{l+m+mi}{1}\PY{p}{]}\PY{o}{*}\PY{o}{*}\PY{l+m+mi}{3} \PY{o}{\PYZhy{}} \PY{n}{x}\PY{p}{[}\PY{l+m+mi}{0}\PY{p}{]}\PY{o}{*}\PY{o}{*}\PY{l+m+mi}{3}\PY{p}{)}\PY{o}{/}\PY{p}{(}\PY{n}{x}\PY{p}{[}\PY{l+m+mi}{1}\PY{p}{]} \PY{o}{\PYZhy{}} \PY{n}{x}\PY{p}{[}\PY{l+m+mi}{0}\PY{p}{]}\PY{p}{)} \PY{o}{\PYZhy{}} \PY{n}{a2}\PY{o}{*}\PY{p}{(}\PY{n}{x}\PY{p}{[}\PY{l+m+mi}{0}\PY{p}{]} \PY{o}{+} \PY{n}{x}\PY{p}{[}\PY{l+m+mi}{1}\PY{p}{]}\PY{p}{)}
         \PY{n}{a0} \PY{o}{=} \PY{n}{f\PYZus{}array}\PY{p}{[}\PY{l+m+mi}{0}\PY{p}{]} \PY{o}{\PYZhy{}} \PY{n}{a1}\PY{o}{*}\PY{n}{x}\PY{p}{[}\PY{l+m+mi}{0}\PY{p}{]} \PY{o}{\PYZhy{}} \PY{n}{a2}\PY{o}{*}\PY{n}{x}\PY{p}{[}\PY{l+m+mi}{0}\PY{p}{]}\PY{o}{*}\PY{o}{*}\PY{l+m+mi}{2} \PY{o}{\PYZhy{}} \PY{n}{a3}\PY{o}{*}\PY{n}{x}\PY{p}{[}\PY{l+m+mi}{0}\PY{p}{]}\PY{o}{*}\PY{o}{*}\PY{l+m+mi}{3}
         \PY{n}{a} \PY{o}{=} \PY{p}{[}\PY{n}{a1}\PY{p}{,} \PY{l+m+mi}{2}\PY{o}{*}\PY{n}{a2}\PY{p}{,} \PY{l+m+mi}{3}\PY{o}{*}\PY{n}{a3}\PY{p}{]} \PY{c+c1}{\PYZsh{}coefficients of f\PYZsq{}}
         
         \PY{c+c1}{\PYZsh{} find the zeros of the f\PYZsq{} polynomial (using the quadratic formula)}
         \PY{n}{b} \PY{o}{=} \PY{n}{a2}\PY{o}{*}\PY{o}{*}\PY{l+m+mi}{2} \PY{o}{\PYZhy{}} \PY{l+m+mi}{3}\PY{o}{*}\PY{n}{a1}\PY{o}{*}\PY{n}{a3}
         \PY{n}{X1} \PY{o}{=} \PY{p}{(}\PY{o}{\PYZhy{}}\PY{n}{a2} \PY{o}{+} \PY{n}{sqrt}\PY{p}{(}\PY{n}{b}\PY{p}{)}\PY{p}{)}\PY{o}{/}\PY{p}{(}\PY{l+m+mi}{3}\PY{o}{*}\PY{n}{a3}\PY{p}{)}
         \PY{n}{X2} \PY{o}{=} \PY{p}{(}\PY{o}{\PYZhy{}}\PY{n}{a2} \PY{o}{\PYZhy{}} \PY{n}{sqrt}\PY{p}{(}\PY{n}{b}\PY{p}{)}\PY{p}{)}\PY{o}{/}\PY{p}{(}\PY{l+m+mi}{3}\PY{o}{*}\PY{n}{a3}\PY{p}{)}
         
         \PY{n+nb}{print}\PY{p}{(}\PY{l+s+s1}{\PYZsq{}}\PY{l+s+s1}{roots = }\PY{l+s+s1}{\PYZsq{}}\PY{p}{,} \PY{n}{X1}\PY{p}{,} \PY{n}{X2}\PY{p}{)}
         
         \PY{c+c1}{\PYZsh{} plot the results}
         \PY{n}{plt}\PY{o}{.}\PY{n}{rc}\PY{p}{(}\PY{l+s+s1}{\PYZsq{}}\PY{l+s+s1}{pdf}\PY{l+s+s1}{\PYZsq{}}\PY{p}{,}\PY{n}{fonttype}\PY{o}{=}\PY{l+m+mi}{3}\PY{p}{)}          \PY{c+c1}{\PYZsh{} for proper subsetting of fonts}
         \PY{n}{plt}\PY{o}{.}\PY{n}{rc}\PY{p}{(}\PY{l+s+s1}{\PYZsq{}}\PY{l+s+s1}{axes}\PY{l+s+s1}{\PYZsq{}}\PY{p}{,}\PY{n}{linewidth}\PY{o}{=}\PY{l+m+mf}{0.5}\PY{p}{)}      \PY{c+c1}{\PYZsh{} thin axes; the default for lines is 1pt}
         \PY{n}{x} \PY{o}{=} \PY{n}{np}\PY{o}{.}\PY{n}{linspace}\PY{p}{(} \PY{l+m+mf}{0.05}\PY{p}{,} \PY{l+m+mf}{0.15}\PY{p}{,} \PY{l+m+mi}{500}\PY{p}{)}
         \PY{n}{plt}\PY{o}{.}\PY{n}{plot}\PY{p}{(}\PY{n}{x}\PY{p}{,} \PY{n}{a0} \PY{o}{+}\PY{n}{a1}\PY{o}{*}\PY{n}{x} \PY{o}{+} \PY{n}{a2}\PY{o}{*}\PY{n}{x}\PY{o}{*}\PY{o}{*}\PY{l+m+mi}{2} \PY{o}{+} \PY{n}{a3}\PY{o}{*}\PY{n}{x}\PY{o}{*}\PY{o}{*}\PY{l+m+mi}{3}\PY{p}{,} \PY{l+s+s1}{\PYZsq{}}\PY{l+s+s1}{k\PYZhy{}\PYZhy{}}\PY{l+s+s1}{\PYZsq{}}\PY{p}{,} \PY{n}{label}\PY{o}{=}\PY{l+s+s1}{\PYZsq{}}\PY{l+s+s1}{Poly. approx.}\PY{l+s+s1}{\PYZsq{}}\PY{p}{)}
         \PY{n}{plt}\PY{o}{.}\PY{n}{plot}\PY{p}{(}\PY{n}{x}\PY{p}{,} \PY{p}{(}\PY{l+m+mf}{1.11} \PY{o}{+} \PY{l+m+mf}{1.11}\PY{o}{*}\PY{n}{x}\PY{o}{*}\PY{o}{*}\PY{p}{(}\PY{o}{\PYZhy{}}\PY{l+m+mf}{0.18}\PY{p}{)}\PY{p}{)}\PY{o}{/}\PY{p}{(}\PY{l+m+mi}{1} \PY{o}{\PYZhy{}} \PY{n}{x}\PY{p}{)}\PY{p}{,} \PY{l+s+s1}{\PYZsq{}}\PY{l+s+s1}{k}\PY{l+s+s1}{\PYZsq{}}\PY{p}{,} \PY{n}{label}\PY{o}{=}\PY{l+s+s1}{\PYZsq{}}\PY{l+s+s1}{Objective func.}\PY{l+s+s1}{\PYZsq{}}\PY{p}{)}
         \PY{n}{plt}\PY{o}{.}\PY{n}{axis}\PY{p}{(}\PY{p}{[}\PY{l+m+mf}{0.05}\PY{p}{,} \PY{l+m+mf}{0.15}\PY{p}{,} \PY{l+m+mf}{3.09}\PY{p}{,}\PY{l+m+mf}{3.17}\PY{p}{]}\PY{p}{)}
         \PY{n}{legend} \PY{o}{=} \PY{n}{plt}\PY{o}{.}\PY{n}{legend}\PY{p}{(}\PY{n}{loc}\PY{o}{=}\PY{l+s+s1}{\PYZsq{}}\PY{l+s+s1}{upper center}\PY{l+s+s1}{\PYZsq{}}\PY{p}{,} \PY{n}{shadow}\PY{o}{=}\PY{k+kc}{False}\PY{p}{,} \PY{n}{fontsize}\PY{o}{=}\PY{l+s+s1}{\PYZsq{}}\PY{l+s+s1}{large}\PY{l+s+s1}{\PYZsq{}}\PY{p}{)}
         \PY{n}{plt}\PY{o}{.}\PY{n}{ylabel}\PY{p}{(}\PY{l+s+s2}{\PYZdq{}}\PY{l+s+s2}{f}\PY{l+s+s2}{\PYZdq{}}\PY{p}{)}
         \PY{n}{plt}\PY{o}{.}\PY{n}{xlabel}\PY{p}{(}\PY{l+s+s2}{\PYZdq{}}\PY{l+s+s2}{x/D}\PY{l+s+s2}{\PYZdq{}}\PY{p}{)}
         \PY{n}{plt}\PY{o}{.}\PY{n}{show}\PY{p}{(}\PY{p}{)}
         
         \PY{n}{poly\PYZus{}root} \PY{o}{=} \PY{p}{[}\PY{l+m+mf}{0.097620387704}\PY{p}{,} \PY{l+m+mf}{0.0985634827486}\PY{p}{,} \PY{l+m+mf}{0.0969340736066}\PY{p}{,} \PYZbs{}
                      \PY{l+m+mf}{0.098775097463}\PY{p}{,} \PY{l+m+mf}{0.102426814371}\PY{p}{,} \PY{l+m+mf}{0.101638472077}\PY{p}{,} \PYZbs{}
                      \PY{l+m+mf}{0.0991941169039}\PY{p}{,} \PY{l+m+mf}{0.095873175811}\PY{p}{]}
         \PY{n+nb}{print}\PY{p}{(}\PY{l+s+s1}{\PYZsq{}}\PY{l+s+s1}{polynomial root std. deviation = }\PY{l+s+s1}{\PYZsq{}}\PY{p}{,} \PY{n}{np}\PY{o}{.}\PY{n}{std}\PY{p}{(}\PY{n}{poly\PYZus{}root}\PY{p}{)}\PY{p}{)}
\end{Verbatim}

    \begin{Verbatim}[commandchars=\\\{\}]
roots =  0.0992038240925 0.283337512729
    \end{Verbatim}

    \begin{center}
    \adjustimage{max size={0.9\linewidth}{0.9\paperheight}}{ME596 Homework 4_files/ME596 Homework 3_6_1.pdf}
    \end{center}
    { \hspace*{\fill} \\}
    
    \begin{Verbatim}[commandchars=\\\{\}]
polynomial root std. deviation =  0.00208605896509
    \end{Verbatim}

    \subsubsection*{Discussion}\label{discussion}

Both the equal interval search and the polynomial approximation seem to
yield results that agree to 3 decimal places. Both also return values
squarely reside within our expected bounds (being between 0.09 and
0.10), which we determined by a qualitative examination of the plot of
the original objective function.

The roots of our polynomial are real and unique. On the bounds
$0.05<x<0.15$ the minimum of the function by the polynomial
approximation is the first root, at $x=0.99$. Though this agrees well
with the result obtained via the equal interval search method for this
problem, the polynomial approximation method appears to be sensitive to
the initial guess points of $x$, even using the relatively accurate
4-point cubic approximation. The standard deviation of ten sets of
randomly sampled $x$ points on our bounds is 0.0021.

The equal interval search might also suffer from choice of an initial
point x, on the objective function, especially if the objective function
is multi-modal. Practically speaking the hole size $d$ must be on the
bounds $0<d<1$ to make physical sense. If $d \geq D$ or $d \leq 0$ the
search function will return divide by zero errors. However, within the
bounds of 0.05 and 0.15 the equal interval search function returns an
average value of 0.0979.


    % Add a bibliography block to the postdoc
    
    
    
    \end{document}
